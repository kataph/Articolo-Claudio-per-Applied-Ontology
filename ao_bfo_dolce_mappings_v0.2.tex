% PLEASE USE THIS FILE AS A TEMPLATE
% Check file iosart2x.tex for more examples
%                   
%%%% Journal title                                          (\documentclass optional parameter)
%%   Applied Ontology                                       (ao)
%%%% IOS Press
%%%% Latex 2e

% add. options: [seceqn,secthm,crcready,onecolumn]

\documentclass[ao]{iosart2x}
\usepackage[T1]{fontenc}
\usepackage{times}%
\usepackage{natbib}
\usepackage{rawfonts}
\usepackage{stmaryrd}
\usepackage{amsmath,txfonts,bbm}
%\usepackage{phonetic} % added to get an upside-down iota (Russell's definite description) \riota.
% incompatibile con \usepackage{mathptmx}
\usepackage{graphicx}
\usepackage{calc}
\usepackage{mathptmx}
\usepackage{graphicx,amssymb,verbatim,color,hyperref}
\usepackage[all]{xy}
\usepackage{multicol}
\usepackage{enumerate}
\usepackage{pdflscape}
\usepackage{float}%exact placement of floats (things inside a begin{})
\usepackage{listings}
\usepackage{nameref}
\lstset{% general command to set parameter(s)
basicstyle=\tiny}%, % print whole listing small
% keywordstyle=\color{black}\bfseries\underbar,
% % underlined bold black keywords
% identifierstyle=, % nothing happens
% commentstyle=\color{white}, % white comments
% stringstyle=\ttfamily, % typewriter type for strings
% showstringspaces=false} % no special string spaces

%\usepackage{textcomp}

%\newlength{\mylength}

% \newcommand{\proofdbth}[2]{#1 - Proof of Theorem \refdbth{#2}}:}
% \newcommand{\proofdolceth}[2]{#1 - Proof of Theorem \refdolceth{#2}}:}
\newcommand{\proofdbth}[2]{\textbf{Proof of Theorem \refdbth{#2}}}
\newcommand{\proofbdth}[1]{\textbf{Proof of Theorem \refbdth{#1}}}
\newcommand{\proofdolceth}[2]{\textbf{Proof of Theorem \refdolceth{#2}}}
%\newcommand{\proofdolceth}[2]{\subsubsection{Proof of Theorem \refdolceth{#2}}}
\newcommand{\counterExampleDB}[1]{\noindent Counter model to \refdbth{#1}}
\newcommand{\counterExampleBD}[1]{\noindent Counter model to \refbdth{#1}}


%\renewcommand{\familydefault}{cmss}
%\fontencoding{TS1}
%\renewcommand{\sfdefault}{pcr}

\newcommand{\nb}[1]{\textcolor{red}{$|$}\marginpar{\hspace*{-0cm}\parbox{20mm}{\scriptsize\raggedright\textcolor{red}{#1}}}}

%\newcommand{\nb}[1]{}

% COUNTERS
% newcommand for list ox formula environment
\newcommand{\bflist}{\begin{list}{}{\setlength{\topsep}{2mm}\setlength{\parsep}{0mm}\setlength{\leftmargin}{9.2mm}\setlength{\labelwidth}{8mm}}}
\newcommand{\eflist}{\end{list}}

\newcommand{\bflistnoind}{\begin{list}{}{\setlength{\topsep}{2mm}\setlength{\parsep}{0mm}\setlength{\leftmargin}{0mm}\setlength{\labelwidth}{8mm}}}
\newcommand{\eflistnoind}{\end{list}}

% newcommand for labels of axioms, definitions and formulae
\newcommand{\bfoAxLabel}{\textrm{a$_\texttt{b}$}}
\newcommand{\bfoDefLabel}{\textrm{d$_\texttt{b}$}}
\newcommand{\bfoFmLabel}{\textrm{f$_\texttt{b}$}}
\newcommand{\bfoThrLabel}{\textrm{t$_\texttt{b}$}}


% newcommand for labels of axioms, definitions and formulae
\newcommand{\dolceAxLabel}{\textrm{a$_\texttt{d}$}}
\newcommand{\dolceDefLabel}{\textrm{d$_\texttt{d}$}}
\newcommand{\dolceFmLabel}{\textrm{f$_\texttt{d}$}}
\newcommand{\dolceThrLabel}{\textrm{t$_\texttt{d}$}}

\newcommand{\dbDefLabel}{\textrm{d$_\texttt{db}$}}
\newcommand{\dbThrLabel}{\textrm{t$_\texttt{db}$}}
\newcommand{\dbAxLabel}{\textrm{a}$_\texttt{db}$}
\newcommand{\dbLemLabel}{\textrm{l$_\texttt{db}$}}

\newcommand{\bdDefLabel}{\textrm{d$_\texttt{bd}$}}
\newcommand{\bdThrLabel}{\textrm{t$_\texttt{bd}$}}
\newcommand{\bdAxLabel}{\textrm{a}$_\texttt{bd}$}


%\newcommand{\dbDefLabel}{\textrm{d\raisebox{-2.5pt}{$\hspace{0.8pt}^\triangleright\hspace{-0.3pt}$}bd}}
%\newcommand{\dbThrLabel}{\textrm{d\raisebox{-2.5pt}{$\hspace{0.8pt}^\triangleright\hspace{-0.3pt}$}bt}}
%\newcommand{\dbAxLabel}{\textrm{d\raisebox{-2.5pt}{$\hspace{0.8pt}^\triangleright\hspace{-0.3pt}$}ba}}


% counter and newcommand for numbering formulas of BFO
\newcounter{cntaxb}
\newcommand{\bfoax}[1]{\refstepcounter{cntaxb}\begin{small}{\bf \bfoAxLabel\thecntaxb\label{#1}}\end{small}}
\newcounter{cntdefb}
\newcommand{\bfodf}[1]{\refstepcounter{cntdefb}\begin{small}{\bf \bfoDefLabel\thecntdefb\label{#1}}\end{small}}
\newcounter{cntfmb}
\newcommand{\bfofm}[1]{\refstepcounter{cntfmb}\begin{small}{\bf \bfoFmLabel\thecntfmb\label{#1}}\end{small}}
\newcounter{cntthrb}
\newcommand{\bfoth}[1]{\refstepcounter{cntthrb}\begin{small}{\bf \bfoThrLabel\thecntthrb\label{#1}}\end{small}}

% counter and newcommand for numbering formulas of DOLCE
\newcounter{cntax}
\newcommand{\dolceax}[1]{\refstepcounter{cntax}\begin{small}{\bf \dolceAxLabel\thecntax\label{#1}}\end{small}}
\newcounter{cntdef}
\newcommand{\dolcedf}[1]{\refstepcounter{cntdef}\begin{small}{\bf \dolceDefLabel\thecntdef\label{#1}}\end{small}}
\newcounter{cntfm}
\newcommand{\dolcefm}[1]{\refstepcounter{cntfm}\begin{small}{\bf \dolceFmLabel\thecntfm\label{#1}}\end{small}}
\newcounter{cntthr}
\newcommand{\dolceth}[1]{\refstepcounter{cntthr}\begin{small}{\bf \dolceThrLabel\thecntthr\label{#1}}\end{small}}

% counter and newcommand for numbering mappings DOLCE => BFO
\newcounter{cntdbdf}
\newcommand{\dbdf}[1]{\refstepcounter{cntdbdf}\begin{small}{\bf \dbDefLabel\thecntdbdf\label{#1}}\end{small}}

\newcounter{cntdbax}
\newcommand{\dbax}[1]{\refstepcounter{cntdbax}\begin{small}{\bf \dbAxLabel\thecntdbax\label{#1}}\end{small}}

\newcounter{cntdbth}
\newcommand{\dbth}[1]{\refstepcounter{cntdbth}\begin{small}{\bf \dbThrLabel\thecntdbth\label{#1}}\end{small}}

\newcounter{cntdblm}
\newcommand{\dblm}[1]{\refstepcounter{cntdblm}\begin{small}{\bf \dbLemLabel\thecntdblm\label{#1}}\end{small}}

% counter and newcommand for numbering mappings BFO => DOLCE
\newcounter{cntbddf}
\newcommand{\bddf}[1]{\refstepcounter{cntbddf}\begin{small}{\bf \bdDefLabel\thecntbddf\label{#1}}\end{small}}

\newcounter{cntbdax}
\newcommand{\bdax}[1]{\refstepcounter{cntbdax}\begin{small}{\bf \bdAxLabel\thecntbdax\label{#1}}\end{small}}

\newcounter{cntbdth}
\newcommand{\bdth}[1]{\refstepcounter{cntbdth}\begin{small}{\bf \bdThrLabel\thecntbdth\label{#1}}\end{small}}


\newcommand{\refdolceax}[1]{({\dolceAxLabel}\ref{#1})}
\newcommand{\refdolcedf}[1]{({\dolceDefLabel}\ref{#1})}
\newcommand{\refdolcefm}[1]{({\dolceFmLabel}\ref{#1})}
\newcommand{\refdolceth}[1]{({\dolceThrLabel}\ref{#1})}


\newcommand{\refbfoax}[1]{({\bfoAxLabel}\ref{#1})}
\newcommand{\refbfodf}[1]{({\bfoDefLabel}\ref{#1})}
\newcommand{\refbfofm}[1]{({\bfoFmLabel}\ref{#1})}
\newcommand{\refbfoth}[1]{({\bfoThrLabel}\ref{#1})}

\newcommand{\refdbax}[1]{({\dbAxLabel}\ref{#1})}
\newcommand{\refdbdf}[1]{({\dbDefLabel}\ref{#1})}
\newcommand{\refdbth}[1]{({\dbThrLabel}\ref{#1})}
\newcommand{\refdblm}[1]{({\dbLemLabel}\ref{#1})}

\newcommand{\refbdax}[1]{({\bdAxLabel}\ref{#1})}
\newcommand{\refbddf}[1]{({\bdDefLabel}\ref{#1})}
\newcommand{\refbdth}[1]{({\bdThrLabel}\ref{#1})}


\newcommand{\ty}[1]{\textsc{#1}}
\newcommand{\pr}[1]{\mathtt{#1}}
\newcommand{\cn}[1]{\mathtt{#1}}
\newcommand{\ifif}{\leftrightarrow}
\newcommand\textequal{%
 \rule[.08ex]{5pt}{0.35pt}\llap{\rule[.78ex]{5pt}{0.35pt}}}
\newcommand{\sdef}{{\hspace{1.5pt}:\hspace{-2.5pt}\textequal\hspace{3pt}}}

\newcommand{\dolce}{{\textsc{dolce}}}
\newcommand{\dolceorig}{{\textsc{dolce-d{\footnotesize 18}}}}
\newcommand{\bfo}{{\textsc{bfo}}}
\newcommand{\bfocl}{{\textsc{bfo-cl}}}
\newcommand{\emmo}{{\textsc{emmo}}}


% THEORIES AND MAPPINGS
\newcommand {\thdolce} {\ensuremath{\mathfrak{D}}}
\newcommand {\thbfo} {\ensuremath{\mathfrak{B}}}
\newcommand {\dbmap} {\ensuremath{\mathfrak{M}_\texttt{db}}}
\newcommand {\bdmap} {\ensuremath{\mathfrak{M}_\texttt{bd}}}
\newcommand {\thbfobdmap} {\ensuremath{\mathfrak{B}_\texttt{d}}}
\newcommand {\thdolcedbmap} {\ensuremath{\mathfrak{D}_\texttt{b}}}




% CATEGORIES OF DOLCE

% Abstract
\newcommand {\ABdcat} {\textsc{ab}}
% Abstract Quality
\newcommand {\AQdcat} {\textsc{aq}}
% Abstract Region
\newcommand {\ARdcat} {\textsc{ar}}
% Achievement
\newcommand {\ACHdcat} {\textsc{ach}}
% Accomplishment
\newcommand {\ACCdcat} {\textsc{acc}}
% Agentive Physical Object
\newcommand {\APOdcat} {\textsc{apo}}
% Agentive Social Object
\newcommand {\ASOdcat} {\textsc{aso}}
% Amount of Matter
\newcommand {\Mdcat} {\textsc{m}}
% Arbitrary Sum
\newcommand {\ASdcat} {\textsc{as}}
% Endurant
\newcommand {\EDdcat} {\textsc{ed}}
% Event
\newcommand {\EVdcat} {\textsc{ev}}
% Feature
\newcommand {\Fdcat} {\textsc{f}}
% Mental Object
\newcommand {\MOBdcat} {\textsc{mob}}
% Non-agentive Physical Object
\newcommand {\NAPOdcat} {\textsc{napo}}
% Non-agentive Social Object
\newcommand {\NASOdcat} {\textsc{naso}}
% Non-physical Endurant
\newcommand {\NPEDdcat} {\textsc{nped}}
% Non-physical Object
\newcommand {\NPOBdcat} {\textsc{npob}}
% Particular
\newcommand {\PTdcat} {\textsc{pt}}
% Perdurant
\newcommand {\PDdcat} {\textsc{pd}}
% Physical Endurant
\newcommand {\PEDdcat} {\textsc{ped}}
% Physical Object
\newcommand {\POBdcat} {\textsc{pob}}
% Physical Quality
\newcommand {\PQdcat} {\textsc{pq}}
% Physical Region
\newcommand {\PRdcat} {\textsc{pr}}
% Process
\newcommand {\PROdcat} {\textsc{pro}}
% Quality
\newcommand {\Qdcat} {\textsc{q}}
% Region
\newcommand {\Rdcat} {\textsc{r}}
% Social Agent
\newcommand {\SAGdcat} {\textsc{sag}}
% Social Object
\newcommand {\SOBdcat} {\textsc{sob}}
% Society
\newcommand {\SCdcat} {\textsc{sc}}
% Space Region
\newcommand {\Sdcat} {\textsc{s}}
% Spatial Location
\newcommand {\SLdcat} {\textsc{sl}}
% State
\newcommand {\STdcat} {\textsc{st}}
% Stative
\newcommand {\STVdcat} {\textsc{stv}}
% Temporal Location
\newcommand {\TLdcat} {\textsc{tl}}
% Temporal Quality
\newcommand {\TQdcat} {\textsc{tq}}
% Temporal Region
\newcommand {\TRdcat} {\textsc{tr}}
% Time Interval
\newcommand {\Tdcat} {\textsc{t}}
%Essential quality
\newcommand {\EQdcat} {\textrm{e}\textsc{q}}


% RELATIONS OF DOLCE

% Rigid
\newcommand {\RGd} {\ensuremath{\pr{RG}}}
% Non-empty
\newcommand {\NEPd} {\ensuremath{\pr{NEP}}}
% Disjoint
\newcommand {\DJd} {\ensuremath{\pr{DJ}}}
% Subsumes
\newcommand {\SBd} {\ensuremath{\pr{SB}}}
% Equal
\newcommand {\EQd} {\ensuremath{\pr{EQ}}}
% Properly Subsumes
\newcommand {\PSBd} {\ensuremath{\pr{PSB}}}
% Leaf
\newcommand {\Ld} {\ensuremath{\pr{L}}}
\newcommand {\SBLd} {\ensuremath{\pr{SBL}}}
\newcommand {\PSBLd} {\ensuremath{\pr{PSBL}}}
\newcommand {\LxXd} {\ensuremath{\pr{L}_X}}
\newcommand {\SBLxXd} {\ensuremath{\pr{SBL}_X}}
\newcommand {\PSBLxXd} {\ensuremath{\pr{PSBL}_X}}
\newcommand {\PTd} {\ensuremath{\pr{PT}}}

\newcommand {\TMPd} {\ensuremath{\pr{TMP}}}

\newcommand {\TPd} {\ensuremath{\pr{tP}}}
\newcommand {\TPPd} {\ensuremath{\pr{tPP}}}
\newcommand {\TOd} {\ensuremath{\pr{tO}}}
\newcommand {\TATd} {\ensuremath{\pr{tAT}}}
\newcommand {\TATPd} {\ensuremath{\pr{tATP}}}
\newcommand {\TSUMd} {\ensuremath{\pr{tSUM}}}
\newcommand {\TDIFd} {\ensuremath{\pr{tSUM}}}
\newcommand {\TPRDd} {\ensuremath{\pr{tPRD}}}

\newcommand {\SUMd} {\ensuremath{\pr{SUM}}}
\newcommand {\DIFd} {\ensuremath{\pr{DIF}}}
\newcommand {\PRDd} {\ensuremath{\pr{PRD}}}

\newcommand {\Pd} {\ensuremath{\pr{P}}}
\newcommand {\PPd} {\ensuremath{\pr{PP}}}
\newcommand {\Od} {\ensuremath{\pr{O}}}
\newcommand {\ATd} {\ensuremath{\pr{AT}}}
\newcommand {\ATPd} {\ensuremath{\pr{ATP}}}
\newcommand {\CPd} {\ensuremath{\pr{CP}}}
\newcommand {\PREd} {\ensuremath{\pr{PRE}}}
\newcommand {\SPREd} {\ensuremath{\pr{sPRE}}}
\newcommand {\DQTd} {\ensuremath{\pr{DQT}}}
\newcommand {\QTd} {\ensuremath{\pr{QT}}}
\newcommand {\QLd} {\ensuremath{\pr{QL}}}
\newcommand {\TQLd} {\ensuremath{\pr{tQL}}}
\newcommand {\QLxTxPDd} {\ensuremath{\pr{QL}_{T,PD}}}
\newcommand {\QLxTxEDd} {\ensuremath{\pr{QL}_{T,ED}}}
\newcommand {\QLxTxTQd} {\ensuremath{\pr{QL}_{T,TQ}}}
\newcommand {\QLxTxPQorAQd} {\ensuremath{\pr{QL}_{T,PQ \vee AQ}}}
\newcommand {\QLxTxQd} {\ensuremath{\pr{QL}_{T,Q}}}
\newcommand {\QLxTd} {\ensuremath{\pr{QL}_{T}}}
\newcommand {\QLxSxPEDd} {\ensuremath{\pr{QL}_{S,PED}}}
\newcommand {\QLxSxPQd} {\ensuremath{\pr{QL}_{S,PQ}}}
\newcommand {\QLxSxPDd} {\ensuremath{\pr{QL}_{S,PD}}}
\newcommand {\QLxSd} {\ensuremath{\pr{QL}_{S}}}

\newcommand {\INxTd} {\ensuremath{\subseteq_T}}
\newcommand {\INPxTd} {\ensuremath{\subset_T}}
\newcommand {\INxSd} {\ensuremath{\subseteq_{S}}}
\newcommand {\INPxSd} {\ensuremath{\subset_{S}}}
\newcommand {\INxSTd} {\ensuremath{\subseteq_{ST}}}
\newcommand {\INxSTxTd} {\ensuremath{\subseteq_{ST}}}
\newcommand {\CNxTd} {\ensuremath{\approx_T}}
\newcommand {\CNxSd} {\ensuremath{\approx_{S}}}
\newcommand {\CNxSTd} {\ensuremath{\approx_{ST}}}
\newcommand {\CNxSTxTd} {\ensuremath{\approx_{ST}}}
\newcommand {\OVxTd} {\ensuremath{\bigcirc_T}}
\newcommand {\OVxSd} {\ensuremath{\bigcirc_{S}}}

\newcommand {\PxTd} {\ensuremath{\pr{P}_T}}
\newcommand {\PxSd} {\ensuremath{\pr{P}_S}}
\newcommand {\NEPxSd} {\ensuremath{\pr{NEP}_S}}
\newcommand {\CMd} {\ensuremath{\pr{CM}}}
\newcommand {\CMNEGd} {\ensuremath{\pr{CM}^\sim}}
\newcommand {\HOMd} {\ensuremath{\pr{HOM}}}
\newcommand {\HOMNEGd} {\ensuremath{\pr{HOM}^\sim}}
\newcommand {\ATOMd} {\ensuremath{\pr{AT}}}
\newcommand {\ATOMNEGd} {\ensuremath{\pr{AT}^\sim}}

\newcommand {\PCd} {\ensuremath{\pr{PC}}}
\newcommand {\PCxCd} {\ensuremath{\pr{PC_C}}}
\newcommand {\PCxTd} {\ensuremath{\pr{PC_T}}}
\newcommand {\MPCd} {\ensuremath{\pr{{mPC}}}}
\newcommand {\MPPCd} {\ensuremath{\pr{mppc}}}
\newcommand {\LFd} {\ensuremath{\pr{lf}}}

\newcommand {\SDd} {\ensuremath{\pr{SD}}}
\newcommand {\GDd} {\ensuremath{\pr{GD}}}
\newcommand {\Dd} {\ensuremath{\pr{D}}}
\newcommand {\ODd} {\ensuremath{\pr{OD}}}
\newcommand {\OSDd} {\ensuremath{\pr{OSD}}}
\newcommand {\OGDd} {\ensuremath{\pr{OGD}}}
\newcommand {\MSDd} {\ensuremath{\pr{MSD}}}
\newcommand {\MGDd} {\ensuremath{\pr{MGD}}}

\newcommand {\SDxSd} {\ensuremath{\pr{SD_S}}}
\newcommand {\PSDxSd} {\ensuremath{\pr{PSD_S}}}
\newcommand {\PinvSDxSd} {\ensuremath{\pr{P^{-1}SD_S}}}
\newcommand {\GDxSd} {\ensuremath{\pr{GD_S}}}
\newcommand {\PGDxSd} {\ensuremath{\pr{PGD_S}}}
\newcommand {\PinvGDxSd} {\ensuremath{\pr{P^{-1}GD_S}}}
\newcommand {\DGDxSd} {\ensuremath{\pr{DGD_S}}}
\newcommand {\SDTxSd} {\ensuremath{\pr{SDt_S}}}
\newcommand {\GDTxSd} {\ensuremath{\pr{GDt_S}}}
\newcommand {\DGDTxSd} {\ensuremath{\pr{DGDt_S}}}
\newcommand {\OSDxSd} {\ensuremath{\pr{OSD_S}}}
\newcommand {\OGDxSd} {\ensuremath{\pr{OGD_S}}}
\newcommand {\MSDxSd} {\ensuremath{\pr{MSD_S}}}
\newcommand {\MGDxSd} {\ensuremath{\pr{MGD_S}}}

\newcommand {\Kd} {\ensuremath{\pr{K}}}
\newcommand {\DKd} {\ensuremath{\pr{DK}}}
\newcommand {\SKd} {\ensuremath{\pr{SK}}}
\newcommand {\GKd} {\ensuremath{\pr{GK}}}
\newcommand {\OSKd} {\ensuremath{\pr{OSK}}}
\newcommand {\OGKd} {\ensuremath{\pr{OGK}}}
\newcommand {\MSKd} {\ensuremath{\pr{MSK}}}
\newcommand {\MGKd} {\ensuremath{\pr{MGK}}}

\newcommand {\EXDd} {\ensuremath{\pr{EXD}}}
\newcommand {\MEXDd} {\ensuremath{\pr{mEXD}}}

\newcommand {\SLCd} {\ensuremath{\pr{SLC}}}
\newcommand {\TLCd} {\ensuremath{\pr{TLC}}}


%================================================================
% BFO categories and relations
%================================================================

\newcommand{\cntbcat}{\cn{cnt}}
\newcommand{\idcntbcat}{\cn{idcnt}}
\newcommand{\gdcntbcat}{\cn{gdcnt}}
\newcommand{\sdcntbcat}{\cn{sdcnt}}
\newcommand{\mtenbcat}{\cn{mten}}
\newcommand{\imenbcat}{\cn{imen}}
\newcommand{\objbcat}{\cn{obj}}
\newcommand{\fobjbcat}{\cn{fobj}}
\newcommand{\objaggbcat}{\cn{objagg}}
\newcommand{\sitebcat}{\cn{site}}
\newcommand{\cfbndbcat}{\cn{cfbnd}}
\newcommand{\sregbcat}{\cn{sreg}}
\newcommand{\rlzenbcat}{\cn{rlzen}}
\newcommand{\occbcat}{\cn{occ}}
\newcommand{\procbcat}{\cn{proc}}
\newcommand{\pbndbcat}{\cn{pbnd}}
\newcommand{\tregbcat}{\cn{treg}}
\newcommand{\stregbcat}{\cn{streg}}
\newcommand{\qltbcat}{\cn{qlt}}
\newcommand{\rqltbcat}{\cn{rqlt}}
\newcommand{\tinstbcat}{\cn{tinst}}
\newcommand{\tintbcat}{\cn{tint}}
\newcommand{\histbcat}{\cn{hist}}
\newcommand{\onetregbcat}{\cn{treg1}}
\newcommand{\zerotregbcat}{\cn{treg0}}
\newcommand{\rolebcat}{\cn{role}}
\newcommand{\dispbcat}{\cn{disp}}
\newcommand{\fntbcat}{\cn{fnt}}
\newcommand{\fptbcat}{\cn{fpt}}
\newcommand{\flnbcat}{\cn{fln}}
\newcommand{\fsfbcat}{\cn{fsf}}
\newcommand{\onesregbcat}{\cn{sreg1}}
\newcommand{\zerosregbcat}{\cn{sreg0}}
\newcommand{\twosregbcat}{\cn{sreg2}}
\newcommand{\threesregbcat}{\cn{sreg3}}





\newcommand{\bfopartic}{\textsc{par}}
\newcommand{\bfouniv}{\textsc{uni}}
\newcommand{\bfoentity}{\textsc{ent}}

\newcommand{\bforigid}{\pr{RG}}
\newcommand{\bfoisa}{\pr{ISA}}
\newcommand{\bfodisj}{\pr{DJ}}

\newcommand{\bfotime}{\textsc{tm}}

\newcommand{\bfocpart}{\pr{cP}}
\newcommand{\bfocppart}{\pr{cPP}}
\newcommand{\bfocdisj}{\pr{cDJ}}
\newcommand{\bfocoverlap}{\pr{cO}}
\newcommand{\bfoopart}{\pr{oP}}
\newcommand{\bfooppart}{\pr{oPP}}
\newcommand{\bfoooverlap}{\pr{oO}}
\newcommand{\bfotpart}{\pr{tmP}}
\newcommand{\bfotppart}{\pr{tmPP}}
\newcommand{\bfotdisj}{\pr{tmDJ}}
\newcommand{\bfotoverlap}{\pr{tmO}}
\newcommand{\bfompart}{\pr{mP}}
\newcommand{\bfoexist}{\pr{EX}}
\newcommand{\bfoiof}[1]{{\,::_{#1\:\!}}}
\newcommand{\bfoinh}{\pr{INH}}
\newcommand{\bfosdep}{\pr{SDEP}}
\newcommand{\bfogdep}{\pr{GDEP}}
\newcommand{\bfooccurs}{\pr{OCCIN}}
\newcommand{\bfolocated}{\pr{LOC}}
\newcommand{\bfosregof}{\pr{SREG}}
\newcommand{\bfosregofocc}{\pr{SREG_O}}
\newcommand{\bfotregof}{\pr{TREG}}
\newcommand{\bfostregof}{\pr{STREG}}
\newcommand{\bfoparticin}{\pr{PTC}}
\newcommand{\bfoconcr}{\pr{CONCR}}
\newcommand{\bforealizes}{\pr{REAL}}
\newcommand{\bfotproj}{\pr{TPROJ}}
\newcommand{\bfosproj}{\pr{SPROJ}}
\newcommand{\bfohistory}{\pr{HIST}}


%==========================================
%==========================================

\pubyear{0000}
\volume{0}
\firstpage{1}
\lastpage{30}

\begin{document}

\begin{frontmatter}

\title{{\bfo}-{\dolce} mappings: an experiment}
\runningtitle{{\bfo}-{\dolce} mappings: an experiment}

\author[A]{\inits{C.} \fnms{Claudio} \snm{Masolo}\ead[label=e1]{masolo@loa.istc.cnr.it}
\thanks{Corresponding author. \printead{e1}.}},
\author[B]{\inits{A.} \fnms{Francesco} \snm{Compagno}\ead[label=e2]{francesco.compagno@unitn.it}}
and
\author[C]{...}
\runningauthor{Masolo et al.}
\address[A]{Laboratory for Applied Ontology,  \institution{ISTC-CNR}, \cny{Italy}\printead[presep={\\}]{e1}}
\address[B]{Computer Science Department, \institution{University of Trento},\cny{Italy}\printead[presep={\\}]{e2}}
\address[C]{??, \institution{??},
\cny{Italy}}
%\printead[presep={\\}]{e3}}

%\author[A]{Claudio Masolo}
%\author[2]{Alessander Botti Benevides}
%\author[3]{Daniele Porello} 
%
%\affil[1]{\small Laboratory for Applied Ontology, ISTC-CNR, Italy}
%\affil[2]{\small Federal University of Espirito Santo, Brazil}
%\affil[3]{\small Free University of Bozen/Bolzano, Italy}

%\author{\name Alessander Botti Benevides \email benevides@loa.istc.cnr.it\\
%       \name Claudio Masolo \email masolo@loa.istc.cnr.it \\
%       \addr Laboratory for Applied Ontology, ISTC-CNR, Trento, Italy}
%
% For research notes, remove the comment character in the line below.
% \researchnote

%==================================================
%==================================================

%==================================================
\begin{abstract}
\noindent 
add abstract
\end{abstract}
%==========================================

\begin{keyword}
\kwd{Ontology}
\kwd{{\bfo}}
\kwd{{\dolce}}
\kwd{Ontology Mappings}
\end{keyword}

\end{frontmatter}

%===================================
\section{Introduction}\label{sect_dolce}
%===================================

{\color{red} **** direi che si tratta di un mapping fatto in OntoCommons con un'idea pluralistica e che qui riportiamo quali sono stati i challenges incontrati e come li abbiamo affrontati oltre poi a dire qualche cosa su alcune differenze trovate tra bfo e dolce / parlerei fin dall'inizio del fatto che abbiamo cercato di usare dei theorem provers / model generator (con successo parziale)}

------

The vision of the OntoCommons project to foster the use of ontology in the NMBP areas focuses on a network of formal ontologies, called Ontology Commons EcoSystem (OCES). 

The idea is to have a pluralistic approach where different ontologies representing different points of views coexist in a given network (partially) aligned via mappings. This network of ontologies is not existent today. The pluralistic approach involves all the levels of ontologies: top-, middle- and domain-ontologies. Among top-level ontologies OntoCommons consider as starting point $\bfo$, $\dolce$ and $\emmo$.

We report on the $\bfo$-$\dolce$ mapping, in particular we consider the most recent Common Logic (CL) versions of these two ontologies collected from the groups that developed them, see Sect.~\ref{sect_bfo_and_dolce} for more details. The choice to use CL is the result of several considerations including 
%the fact 
that CL is a consolidated standard (\url{https://www.iso.org/standard/66249.html}) and that the visibility and stability of standards increases users' trust in the OCES system.

The aims of this paper are:  $(i)$ show what are the theoretical and practical challenges one faces in aligning big and complex (more than 100 FOL axioms) ontologies; $(ii)$ make evident what are the most delicate choices one in confronted with during the alignment process;  and $(iii)$ provide an analysis of the main differences/similarities between $\bfo$ and $\dolce$.



%===================================
\section{{\bfo} and {\dolce}}\label{sect_bfo_and_dolce}
%===================================

%===================================
\subsection{CL version of {\bfo} 2020 released the 12 Nov 2021}\label{sect_bfo}
%===================================

%This section reports the axioms of {\bfo} that are relevant for the mappings from {\dolce} to {\bfo}. 
%%mapping. 
%In Sect.~\ref{sect_mappings_d2b} we discuss the fact that some notions of {\bfo} are not definable in {\dolce}. It is important to observe that for the {\bfo} to {\dolce} mapping we consider the full {\bfo}, see in particular the material in Sec.~\ref{sect_check_dolce_preservation}. 
%
%\medskip
With $\thbfo$ we indicate the logical theory consisting of all the axioms in the CL-version of BFO 2020 / 12 November 2021 (named here {\bfocl}) available from XXXX\nb{add sito web} in Prover9 syntax with four differences: $(i)$ inverse relations in {\bfocl} (e.g., hasContinuantPart is the inverse of continuantPartOf) are captured in $\thbfo$ by inverting the order of arguments (except for the temporal argument);  $(ii)$ the `AtSomeTime' and `AtAllTimes' relations introduced in {\bfocl}, but never used in other axioms of {\bfocl}, are not considered in $\thbfo$;\footnote{Inverse, `AtSomeTime', and `AtAllTimes' relations are used and useful in the OWL version of {\bfo}.} $(iii)$ relations introduced in {\bfocl} with `if and only if' clauses are here introduced via syntactic definitions;
%: \refbfodf{cppart} for continuantProperPartOf, \refbfodf{oppart} for occurrentProperPartOf, \refbfodf{tppart} for temporalProperPartOf, and \refbfodf{bfoinh} for inheresIn.\nb{CM: si può togliere questo credo} 
and $(iv)$ some syntactic definitions are added to improve the readability of formulas.

Since the names of primitives tend to be long, to have more compact formulas, we adopt here the predicates listed in Table~\ref{table_prim_bfo} while for the universals we adopt the individual constants in Table~\ref{table_cat_bfo}.\nb{CM: aggiungere nelle tabelle le primitive che si trovano nel file CL} The taxonomy of $\thbfo$ is depicted in Figure~\ref{figure_tax_bfo} (vertical lines represent ISA relationships, when solid they indicate a partition). Notice that all the universals in the taxonomy in Figure~\ref{figure_tax_bfo} are `rigid' (in the sense that their instances cannot migrate to another universal) except $\objbcat$, $\fobjbcat$, and $\objaggbcat$. %In addition, all the universals directly subsumed by a given universal cover the whole root and are disjoint except $\objbcat$, $\fobjbcat$, and $\objaggbcat$ for which the disjointness is not explicitly stated.

%Third, for relevant relations introduced in {\bfo}-\textsc{cl} with `if and only if' clauses, we introduce corresponding syntactic definitions: \refbfodf{cppart} for continuantProperPartOf, \refbfodf{oppart} for occurrentProperPartOf, \refbfodf{tppart} for temporalProperPartOf, and \refbfodf{bfoinh} for inheresIn.\nb{CM: si può togliere questo credo} 
%Fourth, to improve the readability of formulas, we introduce some syntactic definitions: \refbfodf{time}, \refbfodf{b_iof_notime}, \refbfodf{def_bfoisa}, \refbfodf{coverlap}, \refbfodf{ooverlap}, \refbfodf{toverlap}.

%Few other relevant axioms are still missing in the comparison.


%this theory  
% the consisting of the axioms  \refbfoax{particORuniv}-\refbfoax{tinst_to_treg} together with the syntactic definitions \refbfodf{time}-\refbfodf{bfoinh} introduced in the Sections \ref{bfo:partANDuniv}-\ref{bfo:taxonomy}.

%%==============================
%\subsection{{\bfo} primitive relations}
%%==============================
%
%Table~\ref{table_prim_bfo} lists the primitive relations of {\bfo}. We excluded from this table:
%\begin{itemize}
%\item the inverse relations (e.g.,  hasContinuantPart is the inverse of continuantPartOf) because in FOL it is enough to invert (some of) the arguments (e.g., $\bfocpart(x,y,t)$ vs. $\bfocpart(y,x,t)$);
%
%\item `AtSomeTime' and `AtAllTimes' relations that can be easily defined in FOL, e.g., 
%$\mathrm{continuantPartOfAtSomeTime}(x,y) \sdef \exists t(\bfocpart(x,y,t))$;
%
%\item some relations that can be syntactically defined: continuantProperPartOf \refbfodf{cppart}, occurrentProperPartOf \refbfodf{oppart}, temporalProperPartOf \refbfodf{tppart}, inheresIn \refbfodf{bfoinh}.
%\end{itemize}

\begin{table*}
\caption{Primitive relations of {\bfo}.}\label{table_prim_bfo}
\begin{tabular}{|l|l|}
\hline
$x \bfoiof{t} u$ & $x$ is an instance of $u$ at time $t$ \\
\hline
$\bfoexist(x,t)$ & $x$ exists at time $t$ \\
\hline
$\bfocpart(x,y,t)$ & $x$ is a continuant part of $y$ at time $t$ \\
\hline
$\bfoopart(x,y)$ & $x$ is an occurrent part of $y$\\
\hline
$\bfompart(x,y)$ & $x$ is a member part of $y$\\
\hline
$\bfotpart(x,y)$ & $x$ is a temporal part of $y$ \\
\hline
$\bfostregof(x,y)$ & $x$ occupies spatiotemporal region $y$\\
\hline
$\bfosregof(x,y,t)$ & $x$ occupies spatial region $y$ at time $t$ \\
\hline
$\bfotregof(x,y)$ & $x$ occupies temporal region $y$\\
\hline
$\bfotproj(x,y)$ & $x$ temporally projects onto $y$\\
\hline
$\bfosproj(x,y,t)$ & $x$ spatially projects onto $y$ at time $t$\\
\hline
$\bfooccurs(x,y)$ & $x$ occurs in $y$\\
\hline
$\bfolocated(x,y,t)$ & $x$ is located in $y$ at time $t$\\
\hline
$\bfosdep(x,y)$ & $x$ specifically depends on $y$\\
\hline
$\bfoconcr(x,y)$ & $x$ concretizes $y$\\
\hline
$\bfogdep(x,y,t)$ & $x$ generically depends on $y$ at time $t$\\
\hline
$\bfoparticin(x,y,t)$ & $x$ participates in $y$ at time $t$\\
\hline
$\bforealizes(x,y)$ & $x$ realizes $y$\\
\hline
$\bfohistory(x,y)$ & $x$ is the history of $y$\\
\hline
$\pr{PREC}(x,y)$ & $x$ precedes $y$\\
\hline
$\pr{FINST}(x,y)$ & $x$ is the first instant of $y$\\
\hline
$\pr{LINST}(x,y)$ & $x$ is the last instant of $y$\\
\hline
$\pr{MBAS}(x,y,t)$ & $x$ is the material basis of $y$ at time $t$\\
\hline
\end{tabular}
\end{table*}

%%==============================
%\subsection{{\bfo} categories}
%%==============================

\begin{table*}
\caption{Categories of {\bfo}.}\label{table_cat_bfo}
\begin{minipage}{0.43\textwidth}
\hspace{30pt}
\begin{tabular}{|p{.14\textwidth}|p{.60\textwidth}|}\hline
$\cfbndbcat$ & Continuant Fiat Boundary\\\hline
$\cntbcat$ & Continuant \\\hline
$\dispbcat$ & Disposition \\\hline
$\flnbcat$ & Fiat Line \\\hline
$\fobjbcat$ & Fiat Object \\\hline
$\fptbcat$ & Fiat Point \\\hline
$\fsfbcat$ & Fiat Surface \\\hline
$\fntbcat$ & Function \\\hline
$\gdcntbcat$ & Generically Dep. Continuant \\\hline
$\histbcat$ & History \\\hline
$\idcntbcat$ & Independent Continuant \\\hline
$\imenbcat$ & Immaterial Entity \\\hline
$\mtenbcat$ & Material Entity \\\hline
$\objbcat$ & Object \\\hline
$\objaggbcat$ & Object Aggregate \\\hline
$\bfopartic$ & Particular \\\hline
$\pbndbcat$ & Process Boundary \\\hline
$\procbcat$ & Process  \\\hline
\end{tabular}
\end{minipage}%
\mbox{}\hfill{}
\begin{minipage}{0.43\textwidth}
\hspace{-30pt}\begin{tabular}{|p{.14\textwidth}|p{.60\textwidth}|}
\hline
$\qltbcat$ & Quality \\\hline
$\qltbcat$ & Relational Quality \\\hline
$\rlzenbcat$ & Realizable Entity \\\hline
$\rolebcat$ & Role \\\hline
$\sdcntbcat$ & Specifically Dep. Continuant \\\hline
$\sitebcat$ & Site \\\hline
$\sregbcat$ & Spatial Region \\\hline
$\zerosregbcat$ & 0d Spatial Region \\\hline
$\onesregbcat$ & 1d  Spatial Region \\\hline
$\twosregbcat$ & 2d Spatial Region \\\hline
$\threesregbcat$ & 3d Spatial Region \\\hline
$\tregbcat$ & Temporal Region \\\hline
$\zerotregbcat$ & 0d Temporal Region \\\hline
$\onetregbcat$ & 1d  Temporal Region \\\hline
$\stregbcat$ & Spatiotemporal Region \\\hline
$\tinstbcat$ & Temporal Instant \\\hline
$\tintbcat$ & Temporal Interval \\\hline
& \\\hline
\end{tabular}
\end{minipage}%
\end{table*}

\begin{figure}
\begin{small}
\hspace{-0pt}\xymatrix@R=10pt@C=0pt{
&&&&&&& \bfopartic \ar@{-}[drr] \ar@{-}[dll] \\
&&&&& \cntbcat \ar@{-}[d] \ar@{-}[dl] \ar@{-}[dr] &&&& \occbcat  \ar@{-}[d] \ar@{-}[dl] \ar@{-}[dr] \ar@{-}[drr]\\
&&&& \idcntbcat \ar@{-}[d] \ar@{-}[dlll] & \gdcntbcat & \sdcntbcat \ar@{-}[d] \ar@{-}[dr] && \procbcat \ar@{--}[d] & \pbndbcat & \tregbcat \ar@{-}[d] \ar@{-}[dr]& \stregbcat  \\
& \mtenbcat \ar@{--}[d] \ar@{--}[dl] \ar@{--}[dr]& & & \imenbcat \ar@{-}[d] \ar@{-}[dl] \ar@{-}[dr] & & \qltbcat \ar@{--}[d] & \rlzenbcat \ar@{-}[d] \ar@{-}[dr]& \histbcat & & \zerotregbcat \ar@{--}[d]  & \onetregbcat \ar@{--}[d] \\
\objaggbcat &  \fobjbcat & \objbcat& \cfbndbcat \ar@{-}[d] \ar@{-}[dl] \ar@{-}[dll] &  \sitebcat & \sregbcat \ar@{-}[d] \ar@{-}[dl] \ar@{-}[dr] \ar@{-}[drr] & \rqltbcat & \rolebcat & \dispbcat \ar@{--}[d] & &\tinstbcat & \tintbcat \\
& \fptbcat & \flnbcat & \fsfbcat & \zerosregbcat & \onesregbcat & \twosregbcat & \threesregbcat & \fntbcat \\
}
\end{small}
\caption{Taxonomy of {\bfo} (vertical lines represent ISA relationships, when solid they indicate a partition).}\label{figure_tax_bfo}
\end{figure}


We report below some definitions and axioms to which we will directly refer in the paper. The identifier of the original axiom in {\bfocl} is indicated between square brackets.\nb{CM: forse conviene mettere qui gli assiomi che poi andiamo a considerare nei teoremi invece che scrivere i teoremi per cui basta scrivere $\thdolcedbmap \vdash (\bfoax{xx})$ or $\thdolcedbmap \nvdash (\bfoax{xx})$ o dirlo a parole}
%
\bflist
\item[\bfodf{time}]  $\bfotime(x) \sdef  x \bfoiof{x} \tregbcat$\nb{CM: vedi quali delle seguenti def servono e metti forse qui tutti gli assiomi di bfo che ci servono} 

\item[\bfodf{b_iof_notime}] $x \bfoiof{} u \sdef \exists t(x \bfoiof{t} u)$ 

\item[\bfodf{def_bfoisa}] $\bfoisa(x,y) \sdef \forall zt(z \bfoiof{t} x \to z \bfoiof{t} y)$ 

\item[\bfodf{coverlap}] $\bfocoverlap(x,y,t) \sdef  \exists z(\bfocpart(z,x,t) \land \bfocpart(z,y,t))$

\item[\bfodf{ooverlap}] $\bfoooverlap(x,y) \sdef  \exists z(\bfoopart(z,x) \land \bfoopart(z,y))$ 

\item[\bfodf{toverlap}] $\bfotoverlap(x,y) \sdef \exists z(\bfotpart(z,x) \land \bfotpart(z,y))$
\eflist



%===================================
\subsection{Extension of {\dolce} ISO-CL version}\label{sect_dolce}
%===================================

%%%%%%%%%%%%%%%%%%%%%%%%%%%%%%%%%%%%%%%%%%%%%%%%%%%%%%%%%%%%%%%%%%%%%%%%%%%%%%%%
%DOLCE SIMPLE plus CONCEPTS
%Version for MACE4 / PROVER9 by D. Porello, S. Borgo, L. Vieu.
%Proved consistent.
%
%
%Based on the axioms of DOLCE (D18) proved consistent in
%(v. DolceSimple) https://github.com/spechub/Hets\neglib/blob/master/Ontology/Dolce/DolceSimpl.dol
%NOTE: The names of the axioms and theorems of DOLCE SIMPLE are those from DOLCE D18 for direct comparison.
%(http://www.loa.istc.cnr.it/wp\negcontent/uploads/2020/03/D18inv.31\neg12\neg03.pdf)
%
With $\thdolce$ we indicate the logical theory consisting of all the axioms in the CL-version of {\dolce} available from XXXX\nb{add sito web} in Prover9 syntax. With respect to the original version of {\dolce} introduced in \cite{D18}\nb{CM: cita deliverable} (here named {\dolceorig}) and following the {\dolce-\textsc{iso}} version\nb{CM: non so se vogliamo fare riferimento alla versione iso o meno}, $\thdolce$ has two main simplifications:
\bflist
\item[(1)] Modality operators are not available in the language of Common Logic.\\ 
Formal consequences: {\dolceorig} is a first order {\em modal} theory relying on the modal logic QS5 with constant domain, without modal operators the intended models of {\dolce} change; 
\item[(2)] The mereological fusion operator is not available in the language of Common Logic.\\ 
Formal consequences: given a property expressible in the theory,  {\dolceorig} assumes the existence of the mereological sum of all the entities that satisfy this property, this kind of entity cannot be ensured to exist in logics without the fusion operator.
\eflist

The first simplification dramatically weakens several notions of dependence that are strongly grounded on modality. To partially overcome this problem, $\thdolce$ has the additional primitive of temporary existential dependence ($\EXDd$) that, however, differs from the existential dependence in  {\dolceorig} in several aspects. % as we will see in Section \ref{sect:dolce_dependence} .
% (specific dependence ($\SDd$) is introduced only considering time, see \refdolcedf{dfSDd});

Concerning the second simplification, to partially overcome the lack of the fusion operation $\thdolce$ considers just the axiom of strong supplementation for $\Pd$ (parthood simpliciter) and $\TPd$ (temporary parthood) obtaining a theory based on extensional mereologies (EM) rather than general extensional mereologies (GEM) as in the case of {\dolceorig}. 
Notice that axioms guaranteeing the closure of the domain under binary sums and products are not included in $\thdolce$. The user is free to add the sum and product operators whenever needed. Second, some important notions defined in {\dolceorig} by using mereological fusion are introduced in $\thdolce$ as primitives; in particular, this applies to $\TLCd(x,t)$, read as ``$x$ is (exactly) located at time $t$'', and $\SLCd(x,s,t)$, ``at time $t$, $x$ is (exactly) located at space $s$''.

Following {\dolce-\textsc{iso}}, two additional simplifications are adopted in $\thdolce$:  
%
\bflist
%\item (Ad9) and (Ad15) in D18 are weakened assuming the existence of {binary} sums instead of fusions;
%\item the predicate $\PREd$ (being present) defined {\dolce} by means of the mereological fusion is here introduced as a primitive relation;
\item[--] only direct qualities are considered ($\DQTd$);
\item[--]	(Ad56), (Ad57), (Ad63), and (Ad64) in {\dolceorig} are instantiated only by temporal locations ($\TLdcat$) and time intervals ($\Tdcat$) and by spatial locations ($\SLdcat$) and space regions ($\Sdcat$), i.e., all the quality leaves explicitly introduced in {\dolceorig}.\nb{CM: non sono sicuro farei riferimento agli assiomi di dolce-d18 qui}
%\item {\color{red} the ``spatial inclusion'' relation is not defined here (originally it needs fusion) therefore axioms (Ad19),(Ad28), and (Ad68) are not expressed.}\nb{CM: per questi in effetti si potrebbe fare qualche cosa}

%\item To obtain more populated models, we omit the existence of sums and (Ad29)
\eflist
%DOLCE SIMPLE plus CONCEPTS with respect to DOLCE D18:

%1. Adjunction of the theory of concepts and roles as endurtants (non\negagentive social objects) and of the relation of classification from:

%Claudio Masolo, Laure Vieu, Emanuele Bottazzi, Carola Catenacci, Roberta Ferrario, Aldo Gangemi,and Nicola Guarino.
%Social roles and their descriptions.
%In Proceedings of the 9th International Conference on the Principles of KnowledgeRepresentation and Reasoning (KR\neg2004), pages 267–277, 2004

%In the following, with {\dolce} we indicate the logical theory $\thdolce$ consisting of the axioms  \refdolceax{PdArg}-\refdolceax{disj_tr_ar} together with the syntactic definitions \refdolcedf{def_PPd}-\refdolcedf{dfSPREd} introduced in the Sections \ref{dolce_mereology}-\ref{dolce:taxonomy}. 
Table~\ref{table_prim_dolce} lists the primitive relations of {$\thdolce$}, Table~\ref{table_cat_dolce} lists the categories  of {$\thdolce$}, and Figure~\ref{fig_tax_dolce} shows the taxonomy of {$\thdolce$} (vertical lines represent ISA relationships, when solid they indicate a partition). 

As done for $\thbfo$, we report below some definitions and axioms in $\thdolce$ to which we will directly refer in the paper.\nb{CM: forse conviene mettere qui tutti gli assiomi/def di dolce}

{\color{red} **** add here definitions and axioms of dolce}

%

%%==============================
%\subsection{{\dolce} primitive relations}
%%==============================
\begin{table*}
\caption{Primitive relations of {\dolce}.}\label{table_prim_dolce}
\begin{tabular}{|l|l|}
\hline
$\DQTd(x,y)$ & $x$ is a direct quality of $y$ \\
\hline
$\EXDd(x,y,t)$ & $x$ is existentially dependent on $y$ at time $t$\\
\hline
$\Kd(x,y,t)$ & $x$ constitutes $y$  at time $t$ \\
\hline
$\Pd(x,y)$ & $x$ is part of $y$ \\
\hline
$\PCd(x,y,t)$ & $x$ participates in $y$ at time $t$\\
\hline
$\QLd(x,y)$ & $x$ is the immediate quale of $y$ \\
\hline
$\SLCd(x,y,t)$ & $x$ is (exactly) located at space $y$ at time $t$\\
\hline
$\TLCd(x,t)$ & $x$ is (exactly) located at time $t$ \\
\hline
$\TPd(x,y,t)$ & $x$ is part of $y$ at time $t$ \\
\hline
$\TQLd(x,y,t)$ & $x$ is the temporary quale of $y$ at time $t$\\
\hline
\end{tabular}
\end{table*}

%%==============================
%\subsection{{\dolce} categories}
%%==============================

\begin{table*}
\caption{Categories of {\dolce}.}\label{table_cat_dolce}
\begin{minipage}{0.45\textwidth}
\hspace{30pt}\begin{tabular}{|p{.10\textwidth}|p{.60\textwidth}|}
\hline
$\ABdcat$ & Abstract \\
\hline
$\ACCdcat$ & Accomplishment \\
\hline
$\ACHdcat$ & Achievement \\
\hline
$\APOdcat$ & Agentive Physical Object \\
\hline
$\AQdcat$ & Abstract Quality\\
\hline
$\ARdcat$ & Abstract Region\\
\hline
$\ASdcat$ & Arbitrary Sum \\
\hline
$\ASOdcat$ & Agentive Social Object \\
\hline
$\EDdcat$ & Endurant \\
\hline
$\EVdcat$ & Event \\
\hline
$\Fdcat$ & Feature \\
\hline
$\Mdcat$ & Amount Of Matter \\
\hline
$\MOBdcat$ & Mental Object \\
\hline
$\NAPOdcat$ & Non-agentive Physical Object \\
\hline
$\NASOdcat$ & Non-agentive Social Object \\
\hline
$\NPEDdcat$ & Non-physical Endurant \\
\hline
$\NPOBdcat$ & Non-physical Object \\
\hline
$\PDdcat$ & Perdurant \\
\hline
\end{tabular}
\end{minipage}%
\mbox{}\hfill{}
\begin{minipage}{0.45\textwidth}
\hspace{-30pt}\begin{tabular}{|p{.10\textwidth}|p{.60\textwidth}|}
\hline
$\PEDdcat$ & Physical Endurant \\
\hline
$\POBdcat$ & Physical Object \\
\hline
$\PQdcat$ & Physical Quality\\
\hline
$\PRdcat$ & Physical Region\\
\hline
$\PROdcat$ & Process\\
\hline
$\Qdcat$ & Quality \\
\hline
$\Rdcat$ & Region \\
\hline
$\Sdcat$ & Space Region \\
\hline
$\SAGdcat$ & Social Agent \\
\hline
$\SCdcat$ & Society \\
\hline
$\SOBdcat$ & Social Object \\
\hline
$\SLdcat$ & Spatial Location \\
\hline
$\STdcat$ & State \\
\hline
$\STVdcat$ & Stative \\
\hline
$\Tdcat$ & Time Interval \\
\hline
$\TQdcat$ & Temporal Quality \\
\hline
$\TLdcat$ & Temporal Location \\
\hline
$\TRdcat$ & Temporal Region \\
\hline
\end{tabular}
\end{minipage}%
\end{table*}
%Mereology%%%%%%%%%%%%%%%%%%%%%%%%%%%%%%%%%%%%%%%%%%%%%%%%%%%%%%%%%%%%%%%%%%%%%%%%%%%%%%%%%%

\begin{figure}
\begin{small}
\hspace{-4pt}\xymatrix@R=10pt@C=0pt{
&&&&&&& \circ \ar@{-}[dllll] \ar@{-}[dl] \ar@{-}[drrr] \ar@{-}[drrrrrr]\\
&&&  \EDdcat \ar@{-}[d] \ar@{-}[dl] \ar@{-}[dr] & & &  \PDdcat \ar@{--}[dl] \ar@{--}[dr] &&&&  \Qdcat \ar@{-}[d] \ar@{-}[dl] \ar@{-}[dr] &&&  \ABdcat \ar@{--}[d]\\
  & &  \PEDdcat \ar@{--}[d] \ar@{--}[dl] \ar@{--}[dll] & \NPEDdcat \ar@{--}[d] & \ASdcat &  \EVdcat \ar@{--}[d] \ar@{--}[dr] & &  \STVdcat \ar@{--}[d] \ar@{--}[dr] & & \TQdcat \ar@{--}[d] & \PQdcat \ar@{--}[d] & \AQdcat & & \Rdcat \ar@{-}[d] \ar@{-}[dl] \ar@{-}[dr]\\
\Mdcat & \Fdcat & \POBdcat \ar@{-}[d] \ar@{-}[dl]  &  \NPOBdcat \ar@{--}[d] \ar@{--}[dr]  & & \ACHdcat & \ACCdcat & \STdcat & \PROdcat & \TLdcat & \SLdcat & & \TRdcat \ar@{--}[d] & \PRdcat \ar@{--}[d] & \ARdcat \\
& \APOdcat & \NAPOdcat & \MOBdcat & \SOBdcat \ar@{-}[d] \ar@{-}[dl]&&&&&&&&  \Tdcat & \Sdcat\\
&&& \ASOdcat \ar@{--}[d] \ar@{--}[dr] & \NASOdcat \\
&&& \SAGdcat & \SCdcat \\
}
\end{small}
\caption{Taxonomy of {\dolce} (vertical lines represent ISA relationships, when solid they indicate a partition).}\label{fig_tax_dolce}
\end{figure}

\subsection{Notation}
We use expressions ({\rm a}$x$), ({\rm t}$x$) and ({\rm d}$x$) to label axioms, theorems, and syntactic definitions, respectively. More specifically,
\begin{itemize}
\item (\bfoAxLabel $x$), (\bfoThrLabel $x$), (\bfoDefLabel $x$) are used for axioms, theorems, and definitions of {$\thbfo$}; 
\item (\dolceAxLabel $x$), (\dolceThrLabel $x$), (\dolceDefLabel $x$) are used for axioms, theorems, and definitions of {$\thdolce$}; 
\item %(\dbAxLabel $x$), (\dbThrLabel $x$), 
(\dbDefLabel $x$) define %are used for axioms, theorems, and definitions used in 
the mapping  from ${\thdolce}$ to ${\thbfo}$, and
\item %(\bdAxLabel $x$), (\bdThrLabel $x$), 
(\bdDefLabel $x$) define %are used for axioms, theorems, and definitions used in 
the mapping  from ${\thbfo}$ to ${\thdolce}$.  
\end{itemize}


%===================================
\section{General strategy for the alignment of TLOs}\label{sect_methodology}
%===================================

%The purpose of the report is to present the consolidated formal systems of the two covered TLOs and the alignment across these systems. Indirectly, it also serves as a guideline for the alignment of other TLOs that might be interested in joining the ontology network.

%In the previous sections we have introduced the Common Logic versions of {\bfo} and {\dolce}. 
In this section we introduce the assumptions, the general strategy  and related considerations that we follow to develop the alignment and verify its ``correctness''.

Once the CL-versions of the two ontologies has been collected: % report was developed following these steps:
%
\begin{enumerate}
%\item The versions in Common Logic of the {\bfo} and {\dolce} ontologies were collected from the groups that developed them. The choice to use Common Logic is the result of several considerations including 
%%the fact 
%that Common Logic is a consolidated standard (\url{https://www.iso.org/standard/66249.html}) and that the visibility and stability of standards increases users' trust in the OCES system.
%\item Ontologies with a stable reference version were compared to the version in Common Logic to highlight relevant changes. This case applies to {\dolce} whose reference version was released in 2003 in first-oder modal logic. The version in Common Logic is an adaptation due to the reduced expressivity of Common Logic.
%\item Formal primitive relations were listed.
\item we analyze the CL-axioms together with the available documentation\nb{CM: citare alemeno il documento di bfo che abbiamo considerato} to ensure proper understanding of the intended interpretations of the primitives of the two ontologies;
 \item we establish and document some methodological choices presented below;
%where 
\item we introduce formal mappings from one ontology to the other (and vice versa);
\item we test, with the help of theorem provers, what the mappings do and do not preserve.
\end{enumerate}


We start from three assumptions:
%
\begin{itemize}
\item[--] ({\bf M1}) The informal correspondences between the presentations of the notions in the two ontologies resulting from the analysis of their documentation and the examples therein should be used to suggest class and relation mappings. 
For instance, {\bfo} \emph{occurrents} (\emph{continuants}) are described and used in a way that is similar to {\dolce} \emph{perdurants} (\emph{endurants}); at first sight, the {\bfo} relation \emph{continuantPartOf} is applied coherently to the use of the {\dolce} relation \emph{temporary parthood}, etc. 


\item[--] ({\bf M2}) The ideal goal of the mapping is to have all the domain of quantification of the source ontology included in the domain of quantification of the target ontology. In other terms, the purpose is to maximize the coverage of the entities in one ontology (called \emph{source} ontology, namely, {$\thdolce$} in the case analyzed in Sect.~\ref{sect_d2b}) which are modeled in the other (called \emph{target} ontology, namely, {$\thbfo$} in Sect.~\ref{sect_d2b}). 
Given this scenario, the study of which axioms of the target ontology hold or not after optimizing the mappings from point (M1) can highlight the differences between the ontological commitments of the two ontologies, at least as formalized in CL, and what are the entities of the source ontology that are problematic to model in, or even incompatible with, the target ontology.   

\item[--] ({\bf M3}) Only mappings that can be formalized in FOL are to be considered. This means that we explore the connection between the entities of the source ontology that can be mapped to into the domain of the target ontology. Meta-modeling techniques that require the application of abstraction processes or set-theoretical (second-order) constructions are not considered. These techniques usually require to enrich the domain of the source ontology with additional entities. This change raises concerns about the actual correspondence between the ontological commitments of the source ontology and that of the theory with the enriched domain.\nb{CM: questo ultimo commento non mi convince, abbiamo cercato di restare a FOL perché: (1) possiamo sperare di farci aiutare dai TPs; (2) sfruttiamo l'idea classica di definitional extension (invece che passare a teoremi di rappresentazione che comunque avrebbero senso)} 
\end{itemize}

Assumptions (M2) and (M3) are interrelated and must be seen in the perspective of an interactive development of mappings across ontologies where the analytical step (M1) is a prerequisite. In Sect.~\ref{sect_analysis_d2b} and Sect.~\ref{sect_analysis_b2d}, we list some alternative mappings that modify (by restricting or expanding the mappings proposed in, respectively, Sect.~\ref{sect_mappings_d2b} and Sect.~\ref{sect_mappings_b2d}) the way some notions of the target ontology are seen in terms of the ones of the source ontology. These alternative mappings are proposed to solve some misalignments raised by the work in Sect.~\ref{sect_d2b} and in Sect.~\ref{sect_b2d}. We will also mention alternative (definitely more complex) constructions which extend the domain of the source ontology to match the domain of the target ontology (possibly allowing to define primitives of the target ontology via these new entities).\nb{CM: qui cosa volevamo dire?} 
%In the context of the OCES framework, the alignment strategy to be exploited is exemplified by the mapping from {\bfo} to {\dolce} and by the mappings from {\dolce} to {\bfo} carried out in Sect.~\ref{sect_d2b} and in Sect.~\ref{sect_b2d}.
 
%Discuss here the general strategy followed in this first draft: (-1) illustrate what we will do technically; (0) `maintain' as much entities as possible from {\dolce}; (1) critical choice of mappings and idea of trying to match has much as possible intuitive correspondence between categories and primitive relations (e.g., $\cntbcat$/$\EDdcat$, $\occbcat$/$\PDdcat$, $\bfoopart$/$\Pd$, $\bfocpart$/$\TPd$); (2) look at which ones of the original {\bfo}-axioms are not valid in the {\dolce}+mappings and try to understand the motivation; (3) individuate alternative mappings (modifying the `imported' entities or the relations) that try to fix some of these dis-alignments.

In order to clarify some technical background on assumption (M3), we discuss two ways to extend a theory via definitions. The goal is to have formulas in the source ontology which match (or best approximate) the concepts in the target ontology. Here we take {$\thdolce$} as the source ontology and {$\thbfo$} as the target ontology. Analogous considerations hold for the other direction. The theory $\thdolce$ is extended with a set of syntactic definitions---the mappings in Sect.~\ref{sect_mappings_d2b}---which inject {$\thbfo$} primitives into {$\thdolce$}, i.e., these definitions capture the way {$\thbfo$} primitives can be `understood' in the {\dolce} perspective, in terms of {$\thdolce$} primitives. 
Syntactic definitions are labelled with the symbol {$\sdef$} and are treated as `parametric macros': we syntactically substitute the \emph{definendum} with the \emph{definiens} by suitably matching the parameters. For instance, suppose we aim to define the {$\thbfo$} primitive of  parthood on occurrents via formula $\bfoopart(x,y) \sdef \phi(x,y)$, where $\phi$ is an expression in the language of {$\thdolce$} containing $x$ and $y$. In this case $x$ and $y$ are treated as parameters. The formula $\forall zw(\bfoopart(z,w) \land \bfoopart(w,z) \to z=w)$ in the language of {$\thbfo$} becomes formula $\forall zw(\phi(z,w) \land \phi(w,z) \to z=w)$ in the language of {$\thdolce$}. Similarly, {$\thbfo$}-formula $\bfoopart(\cn{proc\#1},\cn{proc\#2})$ becomes {$\thdolce$}-formula $\phi(\cn{proc\#1},\cn{proc\#2})$. 

One could decide to include these as new formulas in the source ontology obtaining a theory which is an `extension by definitions'. This is obtained by using biconditional conditions (`if and only if') rather than syntactic definitions as described above. E.g., in the previous example one would add the entire formula $\forall xy(\bfoopart(x,y) \ifif \phi(x,y))$ directly into the source theory. The two approaches are quite similar. The advantage of using syntactic definitions instead of extensions by definitions is that the vocabulary of the theory in which we introduce the syntactic definition does not change. Using the `if and only if' clause one automatically adds the defined predicates to the ontology language (e.g., we would add the {$\thbfo$}-predicate $\bfoopart$ into the {$\thdolce$} vocabulary when using the biconditional in the previous example). In Sect.~\ref{sect_problem_univ} we will see that this latter approach may add new ontological commitments with important consequences in particular on the view of universals.

Once {$\thdolce$} has been extended with these definitions (and possibly further syntactic definitions already present in {$\thbfo$})---we dub $\thdolcedbmap$ such extension---it is possible to check how much of the theory $\thbfo$ is preserved. This is done in Sect.~\ref{sect_check_bfo_preservation} where each axiom in $\thbfo$ expressible\footnote{We will see in Sect.~\ref{sect_mappings_d2b} that this mapping is  only partial.} in $\thdolcedbmap$ is proved or disproved in $\thdolcedbmap$. The analysis in Sect.~\ref{sect_analysis_d2b} of the results obtained in Sect.~\ref{sect_check_bfo_preservation} allows, modulo the introduced mapping, to highlight and understand similarities and divergences between the two top-levels. The mapping in the other direction, i.e., the extension $\thbfobdmap$ with the definitions of {$\thdolce$}-primitives in terms of {$\thbfo$}-primitives, is formalized in Sect.~\ref{sect_b2d}. The combined analysis of the mappings {\dolce}-into-{\bfo} and {\bfo}-into-{\dolce} points out genuine ontological differences as well as problematic aspects of the adopted mapping technique. This analysis opens the possibility to refine the mappings in an iterative process to solve, as far as possible, false cases of agreement and disagreement.

%======================================
\subsection{Use of theorems provers and model builders}
%======================================

We use two theorem provers for first-order logic: Prover9\footnote{\url{https://www.cs.unm.edu/~mccune/prover9/}}  and Vampire\footnote{\url{https://vprover.github.io/}}% [inserire citazioni<---]
, as well as the model builder Mace4\footnote{\url{https://www.cs.unm.edu/~mccune/prover9/}}.
We used both Vampire and Prover9, since Vampire is generally faster, but was sometimes outperformed by Prover9 in our testing. Furthermore, Prover9 proofs are shorter since the Prover9 syntax does not require explicit quantification of variables.

%The provers are used in order to infer a logical consequence from a set of axioms. %Since first-order logic is semi-decidable the provers are not guaranteed to prove or disprove the inference of an axiom from a set of premises so the automated provers cannot be expected to prove or disprove all theorems.

Our methodology is as follows. We translate all axioms and definitions of {$\thdolce$} and $\thbfo$ in the  syntax adopted by the provers (Vampire uses tptp, while Prover9 has its own syntax).
%and we submit this translation together with the translation of the theorem to\nb{CM: qui c'era un problema?}  %used them as antecedents as inputs for 
%the theorem prover. We repeated this procedure for all theorems and the provers returned the proofs listed in Section \ref{sec:dolce_theorems}. 
To prove a theorem of the mappings from {\dolce} to {\bfo}, we add to the translation of $\thdolce$ the translation of the mappings in Sect.\ref{sect_mappings_d2b} (i.e., we translate the whole $\thdolcedbmap$) and we submit the translation of the theorem to the provers. Similarly in the case of the mappings from {\bfo} to {\dolce}.
%the same, but we also added the definitions of {\bfo} and the definitions of the ({\dolce} to {\bfo}) mapping in the input of the provers. In this case, the proofs were more difficult, and t
The provers are not able to automatically prove several theorems. We provide manual proofs for all  non automatically proved theorems and we verify (some of) them by introducing intermediate lemmas until the thesis was proved by the provers. %All verified proofs are reported in Section \ref{db_mappings}.\nb{FC: At this moment not all manual proofs have been verified in this way.} 

To formally verify counterexamples for the non-theorems of the mappings, we produce counterexamples by hand using the syntax of the provers in the following way: we write axioms enforcing the existence of the exact number of constants required by the counterexample, as well as the relations between them. To these axioms we added the axioms of {$\thdolce$} (or  {$\thbfo$}) taxonomy, to simplify our work, since the inclusions between the classes of the taxonomy allow us to only work with taxonomical leaves,\nb{CM: leggera modifica} %the classes that are leaves with respect to the order given by the inclusion relation, 
while the disjunction axioms between the classes allow us to skip some axioms stating that individual constants with different names are different. In this way we obtain, for each counterexample, a theory that exactly describes the counterexample syntactically.

Afterward, we run Mace4 using all these axioms as inputs, to verify that the set of axioms is consistent and obtain explicit semantic descriptions of all counterexamples. Then, for each axiom of {$\thdolce$} (or $\thbfo$), we successfully verify that the axiom can be proved from the theory, thus proving that the (unique) model of the theory is a model of {$\thdolce$} ($\thbfo$). 
Lastly, we check that in the model the theorem that we want to disprove is actually false, but, at this point, this is trivial.

Unfortunately, the obvious way of obtaining counterexamples with Mace4, that is, asking Mace4 to directly find a model of all the axioms of {$\thdolce$} ($\thbfo$) with the addition of the negation of the theorem to be disproven, cannot be implemented, as it requires an excessive computational effort.

%In Section \ref{db_counter_examples} we report the syntactic description of the models and not the semantic one, as it is too long and can be generated from the syntactic description using Mace4. Some models are counterexamples of more than one theorem, in those cases we list all the theorems they are counterexamples of. 
%
%In order to verify the mappings and the counterexamples corresponding to the opposite direction, that is, from {\bfo} to {\dolce}, we proceeded in the same way, and the results are shown in sections \ref{bd_mapping} and \ref{bd_counter_examples}. The only difference is that we used the formal axiomatisation of {\bfo} developed by its authors\footnote{\url{https://github.com/BFO-ontology/BFO-2020}}.

%Notice that the proofs use prefixed notation for the ternary (or binary, when the time argument is absent) predicate of instantiation, in place of the infixed syntax ($\bfoiof{}(x,y,t)$ in place of $x\bfoiof{t}y$).


All the proofs and models can be found in \cite{??}.\nb{CM: mettere documento, il del o un rapporto}

%===================================
\section{Preliminary considerations}\label{sect_prelim_considerations}
%===================================
Before formally introducing the mappings and the subsequent process of axiom verification, we add a few preliminary remarks about two problems related to technical choices made in {$\thbfo$} and {{$\thdolce$}}, and to the different level at which these theories investigate some classes. 
%Problem of the universals and problem of the different resolution of the ontologies in some subdomains.


%%===================================
%\subsection{Preliminary remarks}
%%===================================

%===================================
\subsection{Representation of categories}\label{sect_problem_univ}
%===================================

In  {$\thdolce$} the categories in  Table~\ref{table_cat_dolce} are represented by means of FOL unary predicates. These categories are assumed to correspond to rigid properties, that is, an entity cannot change its taxonomic classification. In other terms, if an entity $x$ belongs to a category, it must belong to that category from the time it starts to exist to the time it ceases to exist (if any). For instance, to state that an entity $x$ is an endurant, one writes $\EDdcat(x)$.  
%TOLTO
%\nb{CM: check if the CL version imposes categories to be non-empty} 

In {$\thbfo$}, universals are in the domain of quantification and a temporary \emph{instance-of} primitive relation (written $\bfoiof{}$) is introduced to represent when a particular is an instance of an universal at a given time. For instance, the fact that ``at time $t$, $x$ is a continuant'' is formally represented as $x \bfoiof{t} \cntbcat$, where $\cntbcat$ corresponds to the universal \emph{being a continuant}.
All the universals considered in Table~\ref{table_cat_bfo} are non-empty and rigid through time %(see Sect.~\ref{bfo_rigidity_univ}) 
with the exception of $\objbcat$, $\objaggbcat$, and $\fobjbcat$. For instance, a material entity could belong to $\objbcat$ for some time, and then to $\objaggbcat$ (or $\fobjbcat$) and change again later, provided at each point in time it is classified by one of these three universals.

These different representational choices raise an initial problem because, according to (M3), one should individuate to which kind of {\dolce} entities the {\bfo} universals correspond (Sect.~\ref{sect_analysis_d2b} adds further considerations on this point). 
We consider only 
%on 
the {$\thbfo$}-universals present in the taxonomy, and start from the subset that can be defined in {$\thdolce$}. 
For each universal $\cn{u}$ we introduce a syntactic definition of form: $x \bfoiof{t} \cn{u} \sdef \phi(x,t)$. Note that here the `parameters' are $x$ and $t$, not $\cn{u}$, which is a constant. This means that, first we 
%are 
do not define the notion of instance-of but we 
%are 
%only defining 
define only the instantiation of a given $\cn{u}$. That is, in general, the definitions of $x \bfoiof{t} \cn{u_1}$ and  $x \bfoiof{t} \cn{u_2}$ could be different. 
Second, as discussed earlier, $\cn{u}$ will not appear in $\phi(x,t)$ and thus will not become an individual in the domain of {$\thdolce$}. If we were to use the biconditional conditions, writing $x \bfoiof{t} \cn{u} \ifif \phi(x,t)$ in the theory, both the predicate $\bfoiof{}$ and the individual constant $\cn{u}$ would be added to the vocabulary of {$\thdolce$}, and this would imply to have some universal in the domain of {$\thdolce$}. One alternative way to avoid this problem is to substitute $\pr{U}(x,t) \sdef \phi(x,t)$ for $x \bfoiof{t} \cn{u} \ifif \phi(x,t)$, i.e., by mapping the needed {\bfo} categories to predicates and the instance-of relation to (logical) predication. %\footnote{This could generate some problems if {\bfo} universals are intensional (vs.~extensional) properties. However this aspect does not emerge from the {\bfo} axioms.}
We adopt the first strategy because it is closer to the formalization in $\thbfo$, and it avoids
%in Sect.~\ref{sect_bfo}, we write $x \bfoiof{t} \cn{u}$ instead of $\pr{U}(x,t)$ (see, for instance, \refdbdf{d2b_pbnd} and \refdbdf{d2b_proc} in Sect.~\ref{sect_mappings_d2b}). However, it must be clear that this is just a notational convention that requires neither 
to include universals in the domain of {$\thdolce$} as well as the relation instance-of among the primitives of {$\thdolce$}.
% (as said, we represent instance-of by means of predication).  

In this perspective, {$\thbfo$} axioms quantifying over universals %(see, especially, Sect.~\ref{bfo:instance+exists}) 
are disregarded by the mappings. Weaker versions of these axioms are introduced by considering the (finite) set $\{\cn{u_1}, \dots, \cn{u_n}\}$ of those universals explicitly used to generate the mappings. For instance, \refbfoax{univ_are_insta} can be substituted by  $\exists xt(x \bfoiof{t} \cn{u_1}) \land \ldots \land \exists x(x \bfoiof{t} \cn{u_n})$.\nb{CM: still some of these axioms are not present in sect. \ref{sect_check_bfo_preservation}}
%
\bflist
\item[\bfoax{univ_are_insta}] $\bfouniv(u) \to \exists xt(x \bfoiof{t} u)$
\eflist

%$x \bfoiof{t} \cn{u} \sdef \phi(x,t)$  (see, for instance, \refdbdf{d2b_pbnd} and \refdbdf{d2b_proc} in Sect.~\ref{sect_mappings_d2b}). Note that $\cn{u}$ is not in $\phi(x,t)$ and universals are not in the {\dolce} domain as maybe the form of the definition could suggest. A different way to look at these definitions is to consider the form $\cn{u}(x,t) \sdef \phi(x,t)$, i.e., they introduce a new predicate for each considered universal. We opted for the first form to be closer to the formalization of {\bfo} in Sect.~\ref{sect_bfo}.
%
% and where $\cn{u}$ is one  (see the relative mappings s of instantiation of these universals like, for instance, the following  \refdbdf{d2b_pbnd} or \refdbdf{d2b_proc}. The universals in the {\bfo}-taxonomy are then represented in {\dolce} by means of corresponding predicates. For instance,  $x \bfoiof{t} \procbcat$ can be intuitively seen in {\dolce} as $\procbcat(x,t)$.   


%===================================
\subsection{Different ontological focus and resolution}\label{sect_diff_resolution}
%===================================

The informal categorical distinctions in {\dolce} and {\bfo} are quite similar. The presentations of {\bfo} occurrents and {\dolce} perdurants are very close and this conclusion is further supported by the fact that both the ontologies introduce a primitive of parthood simpliciter for these entities (this parthood relation is called $\Pd$ in {$\thdolce$} and $\bfoopart$ in {$\thbfo$}). 
Analogously, we notice a similarity between {\bfo} continuants and {\dolce} endurants, given that both ontologies conceive the temporary parthood relation as a primitive ($\TPd$ in {$\thdolce$} and $\bfocpart$ in {$\thbfo$}).
Things are different for temporal/spatial  regions and qualities. At first sight, these categories appear to be similar. Formally, they are classified in different ways in the two ontologies. 
{$\thbfo$} classifies qualities (and, more generally, specific dependent continuants) and spatial regions under continuants while temporal regions are classified under occurrents. In {$\thdolce$}, temporal regions, spatial regions, and qualities are neither endurants nor perdurants.\footnote{We will resume this issue in more detail below including a discussion of spatiotemporal regions.} In particular, temporal regions and spatial regions are abstract entities in {$\thdolce$}. This difference---mainly grounded on the way the distinction between endurants/continuants  and perdurants/occurents is characterized---complicates the comparison and needs to be taken into account in defining the mappings (see for instance the definition \refdbdf{d2b_exist} in $\thdolcedbmap$ of the primitive $\bfoexist$ (\emph{existsAt}) of {$\thbfo$}).      

The way in which the most general categories (endurants/continuants on the one hand, perdurants/occurrents on the other hand) are specialized
diverges considerably in the two ontologies:

\paragraph{On endurants vs.~continuants} In {\dolce} perdurants are specialized mainly on the basis of two notions, both extensively discussed in the linguistic and philosophical literature: \emph{homeomericity} (roughly, the parts of a perdurant of a 
%given 
certain kind are also of the same kind) and \emph{cumulativity} (roughly, the mereological sum of perdurants all of which are of a certain kind is of the same kind, too); in {\bfo} processes are distinguished from process boundaries mainly on the basis of the dimensionality of their temporal locations and on the fact that they are temporal proper parts of other occurrents.

\paragraph{On perdurants vs.~occurrents} The spatial and material dimensions of this kind of entities play a central role in both {\dolce} and {\bfo} (but see Sect.~\ref{sect_analysis_d2b} where subtle differences are discussed). {\dolce} presents a finer taxonomy mainly aimed to develop the taxonomy to cover the notions of agentivity and sociality. {\bfo} is driven by the distinction between fiat vs.~bona fide entities, explicitly introduces the notion of aggregate, and relies on the dimensionality of the spatial (rather than temporal) location to distinguish sites vs continuant fiat boundaries.

\paragraph{On qualities} The branch of the taxonomy for qualities in {\dolce} is detached from that of endurants/continuants, differently from {\bfo}, and is motivated by the \emph{comparability} principle: it makes sense to compare the color of a rose and the color of a vase, it does not make sense to compare the color of a rose and the weight of a vase. Qualities are clustered in terms of maximal comparability, e.g., colors, weights, lengths, shapes, etc. form different quality classes. Spaces of regions have a structure and are associated accordingly, e.g., for colors, the space has regions for red, blue, green, etc. Instead, {\bfo} distinguishes classes of specifically dependent continuants according to the existence, the nature of, and the participants to their realization processes.

\smallskip
These different ways of classifying entities are not incompatible \textit{per se}. For instance, one could consider cumulativity, agentivity, comparability in {\bfo}, and dimensionality of temporal/spatial regions and realization processes in {\dolce}. But, clearly, this approach requires to extend {\bfo} and {\dolce}, a choice that should be pondered carefully. This kind of extension is not exploited in this deliverable. The consequence is that some primitive notions and categories of the target ontology cannot be defined in terms of those in the source ontology. This leads to develop a mapping that focuses on the general notions and categories of the two ontologies.
   





% Formally, both {\bfo} and {\dolce} consider, respectively, temporary parthood for continuants and endurants and parthood simpliciter for occurrents and perdurants.
%


%Second, some of the axioms above can be problematic for the mappings because {\dolce} does not consider universals in the domain of quantification. However, the {\bfo}-taxonomy considers only a small number of universals (and the mappings concern only subset of these). We can then consider the following simplifications.
%%
%\bflist
%\item[--] Instead of \refbfoax{iof_args} and \refbfoax{exist_ifif_iof}, we consider: 
%%
%
%\item[--]  \refbfoax{partic2exist} follows from \refbfoax{cnt_to_exist}, \refbfoax{occ_to_exist}, and \refbfoax{cont+occ_iff_pt}.
%
%
%\item[--] The left to right direction of \refbfoax{exist_ifif_iof} would then follow from \refbfoax{exist_args} and \refbfoax{cont+occ_iff_pt}. 
%
%\item[--] Given \refbfoax{exist_ifif_iof}, \refbfoax{iof_time_dissective} can be substituted by \refbfoth{exist_time_dissective}.


%The following mappings will then consider a modified version of $\thbfo$, namely $(\thbfo \setminus \{$\refbfoax{time}-\refbfoax{iof_time_dissective}$\})\cup \{$\refdbax{cont+occ_iff_pt}-\refdbax{maten2exist_db}$\}$.


%{\color{red} ****add a note to make explicit that we do not consider universals in the domain of quantification and that then we define only $x \bfoiof{t} u$ only for the finite set of universals explicitly considered in the taxonomy of {\bfo}} 

%===================================
\section{The mapping from {\dolce} to {\bfo}}\label{sect_d2b}
%===================================

%In this Section we will analyze how and what categories and primitive relations of {\bfo} can be mapped to the ones of {\dolce}. 

%{\color{red} ****add something to explain the structure of the section}
This section presents the technical results of the establishment of a mapping from {\dolce} to {\bfo}, that is, it shows how to define in {$\thdolce$} the categories and primitive relations of {$\thbfo$}. As discussed earlier, the mapping does not cover all categories and relations.

The first part of this section, Sect. \ref{sect_mappings_d2b}, reports the elements covered by the mapping and the conceptual and formal reasons for the limitation. This part makes technically clear the theoretical and formal barriers for a complete coverage of the ontology when this kind of mappings is exploited. 
%This part 
It ends with the list of primitive relations and the list of categories of {$\thbfo$} which are covered by the mappings followed by the corresponding syntactic definitions in {$\thdolce$}.

The second part of this section, Sect. \ref{sect_check_bfo_preservation}, verifies whether the axioms of {$\thbfo$} hold in {$\thdolce$} when extended with the given syntactic definitions. The axioms are clustered according to the relations or categories they characterize. This part is essentially a list of theorems and their proofs. The proofs have been tested using state of the art theorem provers.

The third and final part, Sect. \ref{sect_analysis_d2b}, provides an in-depth analysis of the achieved results, the limitations and possible alternative strategies.


%===================================
\subsection{Mappings}\label{sect_mappings_d2b}
%===================================
In this section we establish how the {\bfo} notions can be mapped to {\dolce}: according to our previous discussion and point (M3) of Sect. \ref{sect_methodology}, we introduce syntactic definitions of {$\thbfo$} primitives in terms of {$\thdolce$} primitives.
Unfortunately, not all the {$\thbfo$} primitives can be defined in {$\thdolce$} in this way. We clarify the main reasons for this limitation when it happens.\nb{CM: da qui in poi controllare l'uso di $\thdolce$ vs. {\dolce} e lo stesso per bfo} 

%Let us start by analyzing the {\bfo} categories. 
First of all, {$\thdolce$} cannot (without suitable extensions) capture the distinctions grounded on the dimensionality of the instances of a given category. This implies that: 
\begin{enumerate}[$(i)$]
\item all the subcategories of $\sregbcat$ and of $\cfbndbcat$ are ruled out; 
\item the distinction between $\sitebcat$ and $\cfbndbcat$ cannot be captured (sites are three-dimensional while continuant fiat boundaries are two-, one-, or zero-dimensional); 
\item the distinction between $\tintbcat$ and $\tinstbcat$ can be only roughly characterized.
\end{enumerate} 

Concerning $(ii)$ we introduce a definition only for the disjunction of $\sitebcat$ and $\cfbndbcat$ (written $\sitebcat{\cup}\cfbndbcat$), see \refdbdf{d2b_siteUcfbnd}. $\sitebcat{\cup}\cfbndbcat$ is clearly not among the categories in the {\bfo} taxonomy and it might not be acceptable as a {\bfo} universal. However, it is used in the mapping with a purely technical role: $x \bfoiof{t} \sitebcat{\cup}\cfbndbcat$ is a shortcut for $x \bfoiof{t} \sitebcat \lor x \bfoiof{t}\cfbndbcat$. %It can be removed if needed.\nb{SB: check} 
Concerning $(iii)$ we identify temporal instants (intervals) with {\dolce} atomic (non-atomic) time intervals, see \refdbdf{d2b_tinst} and \refdbdf{d2b_tint}. Admittedly, this is a very rough characterization. First, temporal atoms can have the same dimensionality of the times they are part of. Second, in {\bfo} the (finite) sum of zero-dimensional temporal regions is still zero-dimensional while the sum of atomic times is always not atomic, i.e., according to \refdbdf{d2b_tinst} and \refdbdf{d2b_tint}, {it is} an interval. Third, {\bfo} time intervals are convex while {\dolce} non-atomic time intervals are not necessarily convex. Thus, \refdbdf{d2b_tint} seems to approximate the category $\onetregbcat$ better than $\tintbcat$. At the same time, if all the non-atomic entities are instances of  $\onetregbcat$, then $\zerotregbcat$ and $\tinstbcat$ collapse. For these reasons, we omit $\onetregbcat$ and $\zerotregbcat$ and 
%we 
consider only the rough\nb{SB: perch\`e rough? si pu\`o togliere?} definition of $\tinstbcat$ and $\tintbcat$. Note that, in {\bfo}, the convexity of intervals is characterized by means of the primitive of precedence ($\pr{PREC}$). To define $\pr{PREC}$ in {\dolce} one should extend the theory with an order relation defined on time intervals. Similarly for $\pr{FINST}$ (firstInstantOf) and $\pr{LINST}$ (lastInstantOf). We then need to rule out $\pr{PREC}$, $\pr{FINST}$, and $\pr{LINST}$.

As anticipated earlier, following point (M1) of Sect. \ref{sect_methodology} the category of specifically dependent continuants ($\sdcntbcat$) seem to correspond to the {\dolce} category of qualities. However, as observed in Sect.~\ref{sect_diff_resolution}, {\bfo} and {\dolce} distinguish the subclasses of these categories on the basis of different criteria. In {\bfo}, the subcategories of $\sdcntbcat$ are mainly characterized by means of the primitives $\bforealizes$ (realizes) and $\pr{MBAS}$ (materialBasisOf) for which, however, the characterization is missing.\footnote{Some necessary conditions can be collected for $\bforealizes$ but these are not enough to characterize the relation.} As a consequence, $\bforealizes$, $\pr{MBAS}$, and the subcategories of $\sdcntbcat$ are not included in the mapping.

A similar situation holds for the subcategories of material entities ($\mtenbcat$). In particular, objects aggregates are intimately connected to the memberOf ($\bfompart$) relation. None of these notions can be defined without an extension of {\dolce}. It follows that $\bfompart$ and the subcategories of $\mtenbcat$ are not covered by the mapping.

Other cases are more subtle. The primitive historyOf ($\bfohistory$) seems to be naturally defined in {\dolce} as:
%
\bflist
\item[] $\bfohistory(x,y) \sdef \forall z(\Pd(z,x) \ifif \forall t(\PREd(z,t) \to \PCd(y,z,t))$
\eflist
%
However the effectiveness of this definition strongly depends on the existential assumptions on perdurants. For instance, if $y$ just participates to a perdurant $p$ but only during a part $t$ of the temporal  extension of $p$ and the temporal part of $p$ during $t$ does not exist---and {\dolce} does not commit to the existence of all the temporal slices of a perdurant---then $y$ would not have an history. These technical difficulties, and the fact that histories seem to have a marginal role in the {\bfo} ontology, suggest to leave out $\bfohistory$ and $\histbcat$. (This choice might be revised in the finale release of the deliverable.)


Spatiotemporal regions are not explicitly present in {\dolce}. One could introduce them among the regions together with the correspondent qualities. Links to the spatial and temporal projections (corresponding to the {\bfo} $\bfotproj$ and $\bfosproj$) would be also necessary.
Alternatively, violating (M2), spatiotemporal regions could be built (at the semantic level, for instance) as sets of couples $\langle$time interval, space region$\rangle$. 
%
These options could be investigated but, since they are 
%being technically 
technical extensions, we do not consider them here. However, it seems that spatiotemporal regions do not add expressive power so the problem might be less controversial from the technical viewpoint. %everything that can be expressed via spatiotemporal regions can also be expressed taking into account $\TLCd$ together with $\SLCd$. 
Sect.~\ref{occupies_streg} delineates how one can avoid spatiotemporal regions by rewriting {\bfo} axioms concerning spatiotemporal regions/locations in terms of {\dolce} time intervals/$\TLCd$ and space regions/$\SLCd$.
%(a straightforward violation of \refbfoax{univ_are_insta} assuring the non-emptiness of $\stregbcat$) by rewriting {\bfo} axioms concerning spatiotemporal regions/locations in terms of {\dolce} time intervals and space regions, see Section \ref{occupies_streg}. 

Finally, following the discussion in Sect.~\ref{sect_problem_univ}, instance-of ($\bfoiof{}$) is not directly defined. We define in {\dolce} the instantiation of the needed universals.

\smallskip
Summing up, among the {\bfo} notions, in the following we introduce: 
\begin{enumerate}[$(i)$]
\item syntactic definitions for the primitive relations: $\bfoexist$, $\bfocpart$, $\bfoopart$, $\bfotpart$, $\bfosregof$, $\bfotregof$, $\bfooccurs$, $\bfolocated$, $\bfosdep$, $\bfoconcr$, $\bfogdep$, $\bfoparticin$; and 
\item syntactic definitions with form $x \bfoiof{t} \cn{u}$ for each category $\cn{u} \in \{\cntbcat$, $\occbcat$, $\idcntbcat$, $\gdcntbcat$, $\sdcntbcat$, $\mtenbcat$, $\imenbcat$, $\sitebcat{\cup}\cfbndbcat$, $\sregbcat$, $\procbcat$, $\pbndbcat$, $\tregbcat$, $\tinstbcat$, $\tintbcat\}$.
\end{enumerate}

%Let us start by analyzing the {\bfo} primitive relations, see Table~\ref{table_prim_bfo}. The primitive relations precedes, firstInstantOf, and lastInstantOf would require to extend {\dolce} with a structuring relation on time intervals. The relations realizes and materialBasisOf would require {\dolce} to be able at least to individuate among its qualities the ones that are realizable entities. We don't see how this is possible without any extension. 

Below we list these syntactic definitions together with a short informal description. A deeper analysis about these definitions and their impact on the preservation of the axioms of {\bfo} is done in Sect.~\ref{sect_analysis_d2b}.

\bflist
\item[\dbdf{d2b_exist}] $\bfoexist(x,t)\sdef \PREd(x,t) \lor (\Tdcat(x) \land \Pd(t,x)) \lor (\ABdcat(x) \land \neg \Tdcat(x) \land \Tdcat(t))$.

%{\color{red} ****check if in {\bfo} a time $t$ exist only during $t$, see \refbfoax{timeiof2tpart}}
\vspace{1pt}
$\bfoexist$ extends $\PREd$: for $\EDdcat$s, $\PDdcat$s, and $\Qdcat$s, $\bfoexist$ and $\PREd$ coincide but $\bfoexist$ applies also to $\ABdcat$s.  Time intervals $\bfoexist$-exist at every subinterval of themselves while the other abstracts entities $\bfoexist$-exist at every time. This extension try to match the fact that, in {\bfo}, both temporal regions and spatial regions (that seem very close to {\dolce} time intervals and space regions, respectively, see \refdbdf{d2b_treg}, \refdbdf{d2b_sreg}) are in time.

\item[\dbdf{d2b_opart}] $\bfoopart(x,y) \sdef \Pd(x,y) \land ((\PDdcat(x) \land \PDdcat(y)) \lor (\Tdcat(x) \land \Tdcat(y)))$

\vspace{1pt}
$\bfoopart$ restricts $\Pd$ that originally applies to all $\ABdcat$s. In this way we try to preserve the fact that in {\bfo} $\bfoopart$ is defined only on occurrents (that include temporal regions while spatial regions are classified under continuants). 

\item[\dbdf{d2b_tpart}] $\bfotpart(x,y) \sdef \bfoopart(x,y) \land 
\forall z (\bfoopart(z,y) \land \forall t (\bfoexist(z,t) \to \bfoexist(x,t)) \to \bfoopart(z,x))$

\vspace{1pt}
This is the classical definition of temporal part/slice, i.e., $x$ is the maximal part of $y$ during the temporal extension of $x$.

\item[\dbdf{d2b_treg}] $\bfotregof(x,t) \sdef \PDdcat(x) \land \TLCd(x,t)$

\vspace{1pt}
$\bfotregof$ is the restriction of $\TLCd$ to $\PDdcat$s (in {\dolce} $\TLCd$ is defined for all the entities present in time, i.e., also for $\EDdcat$s and $\Qdcat$s).  

%{\color{red} ****old definition
%
%$\bfotregof(x,t) \sdef \PDdcat(x) \land \PREd(x,t) \land \neg \exists t'(\PREd(x,t') \land \PPd(t,t'))$}
 
\item[\dbdf{d2b_pbnd}] $x \bfoiof{t} \pbndbcat \sdef \PDdcat(x) \land \exists y(\bfotppart(x,y)) \land \TLCd(x,t) \land \ATd(t)$

\vspace{1pt}
A process boundary is a temporally atomic perdurant that is a temporal proper part of at least another perdurant. As said, the atomicity of the temporal location of a perdurant is an approximation of its instantaneity. 

%{\color{blue} ****$\pbndbcat$ is rigid: $x \bfoiof{} \pbndbcat \land \bfoexist(x,t) \to x \bfoiof{t} \pbndbcat$
%
%\emph{Proof.} By \refdolceax{TLCd_unicity} the temporal location is unique, therefore $x$ exists at only one time.
%}
%
%{\color{red} ****old definition:
%
%$x \bfoiof{t} \pbndbcat \sdef \PDdcat(x) \land \PREd(x,t) \land \exists y(\bfotppart(x,y)) \land \neg\exists y(\bfotppart(y,x))$}

\item[\dbdf{d2b_proc}] $x \bfoiof{t} \procbcat \sdef \PDdcat(x) \land \PREd(x,t) \land \neg(x \bfoiof{t} \pbndbcat)$

\vspace{1pt}
Processes are perdurants that are not process boundaries (all the perdurants are present at some times \refdolceth{-AB_to_PRE}).

%{\color{blue} ****$\procbcat$ is rigid: $\exists u(x \bfoiof{u} \procbcat) \land \bfoexist(x,t) \to x \bfoiof{t} \procbcat$
%
%\emph{Proof.} From $\exists u(x \bfoiof{u} \procbcat)$ we have $\PDdcat(x)$ and by \refdolceax{d_partictime} and \refdolceax{TLCd_unicity}, $x$ has a temporal location and this temporal location is unique. From the definition of $\bfoexist$ and the disjointness between $\PDdcat$ and $\ABdcat$, $\bfoexist(x,t) \land \PDdcat(x)$ implies $\PREd(x,t)$. $x \bfoiof{u} \procbcat \land \PDdcat(x) \land \TLCd(x,v)$ implies $\neg \ATd(v)$ or $\neg \exists y(\bfotppart(x,y))$. Then $\PDdcat(x) \land \PREd(x,t) \land \TLCd(x,v) \land (\neg \ATd(v) \lor \neg \exists y(\bfotppart(x,y)))$, i.e., $\PDdcat(x) \land \PREd(x,t) \land \neg(x \bfoiof{t} \pbndbcat)$.}

\item[\dbdf{d2b_treg}] $x \bfoiof{t} \tregbcat \sdef \Tdcat(x) \land \bfoexist(x,t)$

\vspace{1pt}
Temporal regions coincide with {\dolce} time intervals.

%{\color{blue} ****$\tregbcat$ is rigid: $\exists u(x \bfoiof{u} \tregbcat) \land \bfoexist(x,t) \to x \bfoiof{t} \tregbcat$
%
%\emph{Proof.} From the definition of $\bfoexist$, $\Tdcat(x)$ implies that $\forall t(\bfoexist(x,t))$, therefore we have $\Tdcat(x) \to \forall t(x \bfoiof{t} \tregbcat)$.
%}
%
%{\color{red} ****in {\bfo} temporal intervals are self-connected but this constraint does not apply in general to one-dimensional temporal regions
%
%****in {\bfo} materials entities are assumed to exist at least at one `one-dimensional temporal region', i.e., there are no `instantaneous' material entities, see \refbfoax{maten2exist}; in {\dolce} we do not have the distinction between zero and one dimensional temporal regions therefore we cannot prove this ({\color{blue} but maybe we can say that they need to exist at one non-atomic interval})}

\item[\dbdf{d2b_occ}] $x \bfoiof{t} \occbcat \sdef x \bfoiof{t} \pbndbcat \lor x \bfoiof{t} \procbcat \lor x \bfoiof{t} \tregbcat$

\vspace{1pt}
As said before, spatiotemporal regions are ruled out (but see Sect.~\ref{occupies_streg}).

%{\color{blue} ****$\tregbcat$ is rigid because $\pbndbcat$, $\procbcat$, and $\tregbcat$ are all rigid. It follows that an occurrent can not migrate from a process to a process boundary, etc.}
%
%{\color{red} ****we are ruling out spatiotemporal regions (or $\stregbcat$ is empty, against \refbfoax{univ_are_insta})}

\item[\dbdf{d2b_tinst}]  $x \bfoiof{t} \tinstbcat \sdef x \bfoiof{t} \tregbcat \land \ATd(x)$

\vspace{1pt}
Temporal instants are atomic time intervals.

%{\color{blue} ****$\tinstbcat$ is rigid because $\tregbcat$ is rigid and $\ATd$ does not depends on time.}
% 
%{\color{red} **(1) temporal instants are elucidated in BFO as `without extent', here we cannot guarantee that; (2) zero-dimensional temporal regions are in general sums of `separated' temporal instants but we do not have a connection relation} 

\item[\dbdf{d2b_tint}] $x \bfoiof{t} \tintbcat \sdef x \bfoiof{t} \tregbcat \land \neg\ATd(x)$

\vspace{1pt}
Temporal intervals are non-atomic time intervals (in {\bfo} they are self-connected but this property cannot be defined in {\dolce}).

%{\color{blue} ****$\tintbcat$ is rigid because $\tregbcat$ is rigid and $\ATd$ does not depends on time.}
%
%{\color{red} **(1) temporal intervals are elucidated in BFO as `self-connected', here we cannot guarantee that; (2) one-dimensional temporal regions are temporal regions that have at least a temporal interval as part but possibly can also have temporal instants as parts, THEREFORE, our definition of $\tintbcat$ seems to capture more one-dimensional temporal regions than temporal intervals} 

\item[\dbdf{d2b_mten}] $x \bfoiof{t} \mtenbcat \sdef \ $\parbox[t]{\textwidth}{$\EDdcat(x) \land 
%\neg \ASdcat(x) \land 
\PREd(x,t) \land \exists u(\PREd(x,u) \land \neg \ATd(u)) \land \\
%\neg\exists y(\SDd(x,y)) \land 
\forall u(\PREd(x,u) \to \exists ysr($\parbox[t]{\textwidth}{$\Mdcat(y) \land \SLCd(x,s,u) \land \SLCd(y,r,u) \land \Pd(r,s)))$}}

\vspace{1pt}
Material entities are non instantaneous (to match $x \bfoiof{} \mtenbcat \to \exists t(t \bfoiof{t} \onetregbcat \land \bfoexist(x,t))$, {\bfo} [zuw-1] axiom) endurants that during their whole life are (at least partially) spatially co-localized with an amount of matter.

%{\color{blue} ****MODIFIED to obtain the rigidity of $\mtenbcat$: all the conditions in the definition hold at every time at which $x$ is present (in addition deleted the condition $\neg \ASdcat(x)$). 
%
%****old definition:
%
%$x \bfoiof{t} \mtenbcat \sdef \ $\parbox[t]{\textwidth}{$\EDdcat(x) \land \neg \ASdcat(x) \land \PREd(x,t) \land \exists u(\PREd(x,u) \land \neg \ATd(u)) \land \\
%%\neg\exists y(\SDd(x,y)) \land 
%\exists ysr($\parbox[t]{\textwidth}{$\Mdcat(y) \land \SLCd(x,s,t) \land \SLCd(y,r,t) \land \Pd(r,s))$}}
%}
%
%%(old def) $\ $ $x \bfoiof{t} \mtenbcat \sdef $\parbox[t]{\textwidth}{$(\PEDdcat(x) \lor \NPOBdcat(x)) \land \PREd(x,t) \land \neg\exists y(\SDd(x,y)) \land \\ \forall u(\PREd(x,u) \to 
%%\exists yzsr($\parbox[t]{\textwidth}{$\Mdcat(z) \land \SLCd(x,s,u) \land \SLCd(y,r,u) \land \Pd(r,s) \land \\ (\EXDd(x,y,u) \lor \TPd(y,x,u)) \land \Kd(z,y,u)))$}}
%
%{\color{red} ****added the presence in a non atomic interval to match  \refbfoax{maten2exist}}
%
%{\color{red} ****remember that not all the $\EDdcat$s necessarily have a spatial location}
%
%%{\color{red} ****$\MOBdcat$s are instances neither of $\idcntbcat$ nor of $\sdcntbcat$ because they are specifically dependent $\APOdcat$s}
%
%%****do we need to add $\TPd(x,y,t) \to \EXDd(y,x,t)$ ?? (in this case we can delete $\TPd(y,x,u)$ in the disjunct in \refdbdf{d2b_mten})\\
%%this move can be problematic because all the wholes become specifically dependent on their constant parts, in this case they would not be under independent continuant of {\bfo}}

\item[\dbdf{d2b_siteUcfbnd}] $x \bfoiof{t} (\sitebcat{\cup}\cfbndbcat) \sdef \Fdcat(x) \land \PREd(x,t) \land 
%\neg\exists y(\SDd(x,y)) \land 
\exists s(\SLCd(x,s,t)) \land \\ 
\mbox{}\hfill{}\forall u(\PREd(x,u) \to \neg\exists ysr(\Mdcat(y) \land \SLCd(x,s,u) \land \SLCd(y,r,u) \land \Pd(r,s)))$

\vspace{1pt}
Both sites and continuant fiat boundaries are {\dolce} features localized in space that during their whole life are never (partially) spatially co-localized with an amount of matter.

%{\color{blue} ****MODIFIED to obtain the rigidity of $\sitebcat{\cup}\cfbndbcat$: all the conditions in the definition hold at every time at which $x$ is present. 
%
%****old definition:
%
%$x \bfoiof{t} (\sitebcat{\cup}\cfbndbcat) \sdef \ $\parbox[t]{\textwidth} {$\Fdcat(x) \land \PREd(x,t) \land 
%%\neg\exists y(\SDd(x,y)) \land 
%\exists s(\SLCd(x,s,t)) \land \\ \neg\exists ysr($\parbox[t]{\textwidth}{$\Mdcat(y) \land \SLCd(x,s,t) \land \SLCd(y,r,t) \land \Pd(r,s))$}}
%}
%
%%(old def) $\ $  $x \bfoiof{t} (\sitebcat{\cup}\cfbndbcat) \sdef \ $\parbox[t]{\textwidth}{$\Fdcat(x) \land \PREd(x,t) \land \\ \forall yu(\TPd(y,x,u) \to \neg\exists z(\Mdcat(z) \land \Kd(z,y,u)))$}
%{\color{red} ****in {\bfo} the distinction between instances of $\sitebcat$ and $\cfbndbcat$ is grounded on the dimensionality of the space they occupy}
%%, the distinction between 0- 1- 2- and 3-dimensional regions of space requires {\dolce} to be extended}

\item[\dbdf{d2b_sreg}] $x \bfoiof{t} \sregbcat \sdef \Sdcat(x) \land \bfoexist(x,t)$

\vspace{1pt}
Spatial regions coincide with {\dolce} space regions.

%{\color{blue} ****$\sregbcat$ is rigid, see the proof for $\tregbcat$}

\item[\dbdf{d2b_imen}] $x \bfoiof{t} \imenbcat \sdef x \bfoiof{t} (\sitebcat{\cup}\cfbndbcat) \lor x \bfoiof{t} \sregbcat$

\item[\dbdf{d2b_idcnt}] $x \bfoiof{t} \idcntbcat \sdef x \bfoiof{t} \mtenbcat \lor x \bfoiof{t} \imenbcat$

\item[\dbdf{d2b_sdcnt}] $x \bfoiof{t} \sdcntbcat \sdef \Qdcat(x) \land \PREd(x,t) \land \exists y(y \bfoiof{t} \idcntbcat \land \neg\Sdcat(y) \land \DQTd(x,y))$

\vspace{1pt}
Specifically dependent continuants are {\dolce} qualities inhering (in the sense of $\DQTd$) in an independent continuant (as defined in \refdbdf{d2b_idcnt}) that is not a spatial region.

%{\color{blue} ****$\sdcntbcat$ is rigid
%
%\emph{Proof}. From the ridigity of $\idcntbcat$, $y \bfoiof{t} \idcntbcat$ implies $y$ is and independent continuant at every time it is present/exists. By \refdolceth{PREd_Q}, when $x$ is present also $y$ is present.
%}
%
%{\color{red} ****{\bfo} does not have qualities of occurrents}
%% (?? and of NPEDs that are not NPOB)}
%
%{\color{red} ****{\dolce} does not investigate the difference btw $\qltbcat$ and $\rlzenbcat$ and do not have relational qualities}
%
%%{\color{red} **** instead of a definition here one could consider an axiom of inclusion}

\item[\dbdf{d2b_concr}] $\bfoconcr(x,y,t) \sdef x \bfoiof{t} \sdcntbcat \land \NPEDdcat(y) \land \neg\exists s(\SLCd(y,s,t))  \land \EXDd(y,x,t)$

\vspace{1pt}
Concretization is a form of {\dolce} existential dependence ($\EXDd$) between a non-physical endurant $y$ that does not have a spatial localization and a specifically dependent continuant (as defined in \refdbdf{d2b_sdcnt}) $x$.

%%\\ \exists z(z \bfoiof{t} \idcntbcat \land \neg \Sdcat(z) \land \DQTd(x,z))$}
%
%{\color{red} ****I think that $\forall t(\PREd(y,t) \to \neg\exists s(\SLCd(y,s,t)))$ follows from $\neg\exists s(\SLCd(y,s,t))$ if we consider the new version of DOLCE}

\item[\dbdf{d2b_gdcnt}] $x \bfoiof{t} \gdcntbcat \sdef \PREd(x,t) \land \forall u(\PREd(x,u) \to \exists y(\bfoconcr(y,x,u)))$

\vspace{1pt}
Generic dependent continuants are non-physical endurants (from \refdbdf{d2b_concr}) that are concretized during their whole life.

%{\color{blue} ****MODIFIED to obtain the rigidity of $\gdcntbcat$: all the conditions in the definition hold at every time at which $x$ is present. 
%
%****old definition:
%
%$x \bfoiof{t} \gdcntbcat \sdef \exists y(\bfoconcr(y,x,t))$
%}
%
%
%{\color{red} old old definition:
%
%$x \bfoiof{t} \gdcntbcat \sdef \NPEDdcat(x) \land \neg \exists s(\SLCd(x,s,t)) \land \\\mbox{} \hfill{} \forall t'(\PREd(x,t') \to \exists y(y \bfoiof{t'} \idcntbcat \land \neg\Sdcat(y) \land \EXDd(x,y,t') \land \EXDd(y,x,t')))$}

\item[\dbdf{d2b_cnt}] $x \bfoiof{t} \cntbcat \sdef x \bfoiof{t} \idcntbcat \lor x \bfoiof{t} \sdcntbcat \lor x \bfoiof{t} \gdcntbcat$

%{\color{red} ****{\dolce} does not directly analyze the nature of $\gdcntbcat$ that however could be considered as a subclass of (non spatially localized) $\NPEDdcat$; according to the BFO-2020 doc the generically dependent continuants are patterns that can be realized in multiple copies specifically dependent on the pattern; \emph{as first tentative approximation} one could define $\gdcntbcat$ as the subclass of $\NPEDdcat$s whose instances are generically dependent on $\idcntbcat \land \neg\sregbcat$ entities that are specific dependent on them:\\\vspace{-10pt}
%
%$x \bfoiof{t} \gdcntbcat \sdef \ $\parbox[t]{\textwidth} {$\NPEDdcat(x) \land \forall t'(\PREd(x,t') \to \\ \exists y(y \bfoiof{t'} \idcntbcat \land \neg(y \bfoiof{t'} \sregbcat) \land \EXDd(x,y,t') \land \EXDd(y,x,t')))$}
%}

\item[\dbdf{d2b_cpart}] $\bfocpart(x,y,t) \sdef x \bfoiof{t} \cntbcat \land y \bfoiof{t} \cntbcat \land (\TPd(x,y,t) \lor \Pd(x,y) \lor \exists zu(\DQTd(x,z) \land \DQTd(y,u) \land \TPd(z,u,t)))$

\vspace{1pt}
According to \refdbdf{d2b_cnt}, \refdbdf{d2b_idcnt}, \refdbdf{d2b_sdcnt}, and \refdbdf{d2b_gdcnt}, continuants include $\EDdcat$s, $\Sdcat$s, and $\Qdcat$s. For $\EDdcat$s, $\bfocpart$ coincides with $\TPd$. For $\PDdcat$s, $\bfocpart$ coincides with $\Pd$. For $\Qdcat$s, in {\dolce}, there is not a parthood relation but we can use the parthood relation among the endurants these qualities inhere in (according to {\dolce} each quality inheres in a single entity,  $\DQTd(x,y) \land \DQTd(x,z) \to y=z$).\footnote{Note that this definitions allows qualities of different kind to be one part of the other (for instance the color of an object is part of the temperature of a bigger object). To avoid these cases one could assume that every specialization of {\dolce} has a (finite) set of leaf-types of qualities and require, in the last disjunct in \refdbdf{d2b_cpart}, $x$ and $y$ to be instances of the same leaf-type. We leave the  implementation and the study of this new definition for future work.}
%; one must check if this move introduces problems with some $\bfocpart$-properties (e.g., extensionality): for instance, some parts of an object with mass could be massless}

%{\color{red} ****{\bfo} continuants include spatial regions (for which $\Pd$ instead of $\TPd$ is considered) and some qualities (for which a parthood relation defined in terms of the one between the continuants they inhere in is considered)}

\item[\dbdf{d2b_sregof}] $\bfosregof(x,s,t) \sdef x \bfoiof{t} \idcntbcat \land \neg\Sdcat(x) \land \SLCd(x,s,t)$

\vspace{1pt}
$\bfosregof$ is the resctriction of $\SLCd$ to independent continuants (as defined in \refdbdf{d2b_idcnt}) that are not spatial regions.  

\item[\dbdf{d2b_partic}] $\bfoparticin(x,y,t) \sdef x \bfoiof{t} \cntbcat \land \neg\Sdcat(x) \land y \bfoiof{t} \procbcat \land \PCd(x,y,t)$

\vspace{1pt}
$\bfoparticin$ is the resctriction of $\PCd$ to continuants that are not spatial regions and to processes (as defined above).  

\item[\dbdf{d2b_occurs}] $\bfooccurs(x,y) \sdef $\parbox[t]{\textwidth} {$\PDdcat(x) \land \exists t(y \bfoiof{t} \mtenbcat \lor y \bfoiof{t} \sitebcat{\cup}\cfbndbcat) \land \\ \forall t(\PREd(x,t) \to \exists sr(\SLCd(x,s,t) \land \SLCd(y,r,t) \land \Pd(s,r)))$}

\vspace{1pt}
$\bfooccurs(x,y)$ holds when the spatial location of the perdurant $x$ is always included in the one of the material entity/site/continuant fiat boundary $y$ (this definition closely corresponds to the one in the documentation of BFO).

%{\color{red} ****this definition closely corresponds to \refbfodf{df_occurs} (doc. BFO 2020 p.101) but it requires  $\exists t(y \bfoiof{t} \mtenbcat \lor y \bfoiof{t} \sitebcat{\cup}\cfbndbcat)$ rather than $ \exists t(y \bfoiof{t} \mtenbcat \lor y \bfoiof{t} \sitebcat)$}

\item[\dbdf{d2b_located}] $\bfolocated(x,y,t) \sdef x \bfoiof{t} \idcntbcat \land \neg\Sdcat(x) \land y \bfoiof{t} \idcntbcat \land \neg\Sdcat(y) \land
\exists sr(\SLCd(x,s,t) \land \SLCd(y,r,t) \land \Pd(s,r))$

\vspace{1pt}
At time $t$, and independent continuant (that is not a space region) is located in another independent continuant (that is not a space region) when the spatial location (at $t$) of the first continuant is included in the spatial location (at $t$) of the second continuant (this definition closely corresponds to the one in the documentation of BFO).

%{\color{red} ****this definition closely corresponds to the one in doc of BFO-2020 p.66}
%%{\color{red} ****it is not clear what the spatial location of a continuant during an extended interval is; in {\dolce} one could strengthen the last conjunct with \\ $\forall t'(\Pd(t',t) \to \exists sr(\SLCd(x,s,t') \land \SLCd(y,r,t') \land \Pd(s,r)))$}
%
%%$\bfolocated(x,y,t) \sdef  \ $\parbox[t]{\textwidth} {$x \bfoiof{t} \idcntbcat \land \neg\Sdcat(x) \land y \bfoiof{t} \idcntbcat \land \neg \Sdcat(y) \land
%%\\ \forall t'(\Pd(t',t) \to \exists sr(\SLCd(x,s,t') \land \SLCd(y,r,t') \land \Pd(s,r)))$}}

\item[\dbdf{d2b_sdep}] $\bfosdep(x,y) \sdef x \bfoiof{} \sdcntbcat \land \DQTd(x,y)$

\vspace{1pt}
$\bfosdep$ is the restriction of $\DQTd$ to specifically dependent continuants (as defined in \refdbdf{d2b_sdcnt}.

%{\color{red}
%
%****old definition:
%
%$\bfosdep(x,y) \sdef \ $\parbox[t]{\textwidth} {$\exists t(x \bfoiof{t} \sdcntbcat) \land \exists t(y \bfoiof{t} \sdcntbcat \lor (y \bfoiof{t} \idcntbcat \land \neg\Sdcat(y))) \land \\ \SDd(x,y) \land \neg \exists t(\bfocoverlap(x,y,t))$}
%}
%
%%{\color{red} ****differently from {dolce}, in {\bfo} specific dependence excludes temporary overlaps, see \refbfoax{sdep2disj}} 
%
%%{\color{red} ****this is a restriction of $\SDd$ defined in \refdolcedf{dfSDd}}

\item[\dbdf{d2b_gdep}] $\bfogdep(x,y,t) \sdef x \bfoiof{t} \gdcntbcat \land y \bfoiof{t} \idcntbcat \land \neg\Sdcat(y) \land \exists zs(\Pd(s,t) \land \DQTd(z,y) \land \bfoconcr(z,x,s))$

\vspace{1pt}
At time $t$, the generic dependent continuant $x$ generically depends on the independent continuant (that is not a spatial region) $y$, when it is concretized (during a part $s$ of $t$) by a specifically dependent continuant $z$ inhering in $y$. 

%{\color{red} ****stronger possibility: $\bfogdep(x,y,t) \sdef \exists z(\DQTd(z,y) \land \bfoconcr(z,x,t))$}
%
%%\item[\dbdf{d2b_gdep}] $\bfogdep(x,y,t) \sdef x \bfoiof{t} \gdcntbcat \land y \bfoiof{t} \idcntbcat \land \neg\Sdcat(y) \land \EXDd(x,y,t) \land \EXDd(y,x,t)$
%%
%%%{\color{red} ****this definition makes sense only if we assume the definition of $\gdcntbcat$ introduced in the discussion of \refdbdf{d2b_cnt}}
%%%****different from the one of dolce, quite peculiar relation, given the poor axiomatization one could try to define in terms of $\EXDd$ (together with realization and concretization, but this would not be very interesting
%%
%%
%%\item[\color{red} (??)] $\bfoconcr(x,y,t) \sdef (x \bfoiof{t} \sdcntbcat \lor  x \bfoiof{t} \procbcat) \land (\exists z(\DQTd(x,z) \land \bfogdep(y,z,t)) \lor \ ????$
%%
%%{\color{red} ****even by considering the definition of $\gdcntbcat$ introduced in the discussion of \refdbdf{d2b_cnt}, the concretization relation probably cannot be defined in the case $x$ is a process}
%%% (in the case $x$ is a specifically dependent continuant one can use what sketched in \refdbdf{d2b_concr})}
%%

\item[\dbdf{d2b_particular}] $\bfopartic(x) \sdef \exists t(\bfoexist(x,t))$

\vspace{1pt}
Particulars are entities that $\bfoexist$-exist in time (to match $\bfoexist(x,t) \to \bfopartic(x) \land \bfotime(t)$ {\bfo}-[oap-1] and $\bfopartic(x) \to \exists t(\bfoexist(x,t))$ {\bfo}-[nmq-1]).

\item[\dbdf{d2b_univ}] $\bfouniv(x) \sdef \neg\bfopartic(x)$

\vspace{1pt}
Universals are non-particulars (to match $\bfopartic(x) \lor \bfouniv(x)$ {\bfo}-[eto-1] and $\bfopartic(x) \to \neg \bfouniv(x)$ [qkp-1]).

%Note that, according to (M3) all these mappings are syntactic definitions.\nb{CM: added, maybe to move before the introduction of mappings\\ SB: lo abbiamo già detto in più punti, io toglierei questa osservazione}

%\item[\color{red} (??)] $\bfompart(x,y,t) \sdef \ ??$
%
%{\color{red} ****the distinction between objects and object aggregates is not analyzed in {\dolce} and seems to require additional expressivity to be captured}
%
%\item[\color{red} (??)] $\bforealizes(x,y) \sdef \ ??$ \hfill{}(but see Sect.~\ref{sect_d2b_realizes} for a partial mapping)
%
%{\color{red} ****the distinction between qualities and realizable entities is not analyzed in {\dolce} and seems to require an extension to be captured, in addition realizability is not clear to us}
\eflist

%=================================
\subsection{Check of the preservation of the original {\bfo} axioms}\label{sect_check_bfo_preservation}
%=================================

As discussed in Sect.~\ref{sect_diff_resolution} (and made explicit in Sect.~\ref{sect_mappings_d2b}) some {\bfo}-axioms involve relations or categories that have not been defined in the mappings. In what follows, when possible and relevant we consider approximations of these axioms as expressible in the {\dolce} language, otherwise we set them aside.

%Let us include \refdbdf{d2b_mten}-\refdbdf{d2b_gdep} into $\dbmap$.
In the rest of this section, $\thdolcedbmap$ indicates the `extension' of $\thdolce$ with the mappings introduced in Sect.~\ref{sect_mappings_d2b} together with the syntactic definitions introduced in Sect.~\ref{sect_bfo}, i.e., formally $\thdolcedbmap = \thdolce \cup \{$\refdbdf{d2b_exist}-\refdbdf{d2b_particular}$\} \cup \{$\refbfodf{time}, \refbfodf{cppart}-\refbfodf{bfoinh}$\}$.

Note that \refbfodf{b_iof_notime} and \refbfodf{def_bfoisa} are not included in $\thdolcedbmap$ because they require universals in the domain of quantification. Moreover, \refbfodf{b_iof_notime} could be avoided by including the existential quantifier on time. Alternatively, for each predicate $\cn{u}$ representing a {\bfo}-category, one could introduce the syntactic definition $x \bfoiof{} \cn{u} \sdef \exists t(x \bfoiof{t} \cn{u})$. Definition \refbfodf{def_bfoisa} is also not needed since it does not occur in the {\bfo} axioms. This definition will be used in the mapping from {\bfo} to {\dolce}, see Sect.~\ref{sect_mappings_b2d}. 

%We succeeded in proving or disproving all the axioms (expressible by the limited vocabulary taken into account in the mappings) reported in Sect.~\ref{sect_bfo} with the exception of \refdbth{d2b_cpart_ment}.

Several proofs has been verified by using theorems provers, they are reported in Sect.~\ref{db_mappings}.


%===================================
\subsection{Analysis}\label{sect_analysis_d2b}
%===================================

We begin this analysis by considering how the entities in the domain of {\dolce} are classified from the point of view of {\bfo}. \refdbth{-AB_particulars} shows that endurants, perdurants, and qualities are all {\bfo} particulars (in the sense of the {\bfo} notion of particular, $\bfopartic$). The case of abstracts is slightly more complex. \refdbth{all_particulars} shows that if there exists at least a time interval then all the abstracts are {\bfo} particulars too. This is due to the definition of $\bfoexist$ \refdbdf{d2b_exist} that assures that abstracts exist at every time. However, while in  {\dolce} the existence of endurants, perdurants, and qualities requires them to have a temporal location (and then to be present in time), see \refdolceax{d_partictime} and \refdolceth{-AB_to_PRE}, this does not hold for abstracts, i.e., it is possible to have models of {\dolce} that contain only abstracts and no time intervals. In this case we would not have {\bfo} particulars, only universals \refdbth{no_particular}.

\bflist
\item[\dolceax{d_partictime}] $(\EDdcat(x) \lor \PDdcat(x) \lor \Qdcat(x)) \to \exists t(\TLCd(x,t))$ 

\item[\dolceth{-AB_to_PRE}] $\thdolce \vdash (\EDdcat(x) \lor \PDdcat(x) \lor \Qdcat(x)) \to \exists t(\PREd(x,t))$

\item[\dbth{-AB_particulars}] $\thdolcedbmap \vdash (\EDdcat(x) \lor \PDdcat(x) \lor \Qdcat(x)) \to \bfopartic(x)$

\item[\dbth{all_particulars}] $\thdolcedbmap \vdash \exists u(\Tdcat(u)) \to \forall x(\bfopartic(x))$

\item[\dbth{no_particular}] $\thdolcedbmap \vdash \neg\exists u(\Tdcat(u)) \to \forall x(\bfouniv(x))$

\eflist

\refdolceth{-AB_to_PRE} seems to confirm both (M1) and (M2), i.e., the general idea that {\dolce} particulars roughly correspond to {\bfo} particulars and that all the particulars of {\dolce} are `imported' into the ones of {\bfo}. However, \refdbth{no_particular} introduces some doubts about the nature of abstracts. The definition of $\bfoexist$ \refdbdf{d2b_exist}, starting from the fact that temporal and spatial regions are particulars in {\bfo}, extended this idea to all regions (and to all abstracts), i.e., there is a presupposition that all the regions have a uniform nature, they are all particulars in this case (even though, in {\bfo}, temporal and spatiotemporal regions are occurrents while spatial regions are continuants). However, other options are possible. One could consider, for instance, the following two variants of \refdbdf{d2b_exist}:
\bflist
\item[--] $\bfoexist(x,t)\sdef \PREd(x,t) \lor (\Tdcat(x) \land \Pd(t,x)) \lor (\Sdcat(x) \land \Tdcat(t))$
\item[--] $\bfoexist(x,t)\sdef \PREd(x,t)$
\eflist
%
In the first case time intervals and space regions become {\bfo} particulars as before (matching the fact that temporal and spatial regions are particulars in {\bfo}) but, according to \refdbdf{d2b_particular} and \refdbdf{d2b_univ}, the rest of abstracts would be included as universals. This possibility is interesting because the examples for {\dolce} (spaces of) regions are colors, weights, shapes, etc., while, in {\bfo}, \emph{being red}, \emph{being blue}, \emph{being round}, etc., are universals. In this perspective, one can also presuppose that the $\Pd$ relation holding between regions (in a given space) represents a sort of (intensional) ISA relation, while $\TQLd$---or, maybe better, the composition of $\DQTd$ and  $\TQLd$---represents instantiation (at a time). An option that deserves a detailed analysis even though it introduces differentiation between time intervals and space regions on the one side, and the remaining regions on the other side. 

The second option excludes all abstracts, including time intervals and space regions, from particulars. Even in this case one could see $\Pd$ as a sort of ISA relation and $\TLCd$ and $\SLCd$ as sorts of instantiation relations (here note that the \refdolceax{TLCd_PD} and  \refdolceax{SLCd_PEDs} go in the direction of considering $\TLCd$ and $\SLCd$ as compositions of $\DQTd$ and  $\TQLd$ as suggested in the previous case also).

In both these cases, the fact that (some) regions become universals is grounded on  \refdbdf{d2b_particular} and \refdbdf{d2b_univ}. The latter try to match \refbfoax{particORuniv} and \refbfoax{particNOuniv}, i.e., the fact that the entities in the domain of {\bfo} are partitioned into universals and particulars. One could, however, decide to split \refbfoax{particORuniv} and assume that regions are neither {\bfo} particulars nor {\bfo} universals. Basically, they would form a kind of entity that {\bfo} does not consider. Also this option deserves a deeper analysis.
%
\bflist
\item[\dolceax{TLCd_PD}] $\PDdcat(x) \to (\TLCd(x,t) \ifif \exists y(\DQTd(y,x) \land \QLd(t,y) \land \Tdcat(t)))$ 
\item[\dolceax{SLCd_PEDs}] $\PEDdcat(x) \to (\SLCd(x,s,t) \ifif \exists y(\DQTd(y,x) \land \SLdcat(y) \land \TQLd(s,y,t)))$ 

\item[\bfoax{particORuniv}] $\bfopartic(x) \lor \bfouniv(x)$
\hfill {\scriptsize [eto-1]}

\item[\bfoax{particNOuniv}] $\bfopartic(x) \to \neg \bfouniv(x)$
\hfill {\scriptsize [qkp-1]}
\eflist

Moving to finer correspondences, \refdbth{pd_to_proc_or_pbnd} shows that perdurants are the union of processes and process boundaries. However, \refdbth{d2b_procdissect}, \refdbth{d2b_opart_proc3}, \refdbth{d2b_opart_pbnd2}, \refdbth{d2b_pbndifif}, and \refdbth{d2b_proctreg2tint} make evident that the distinction between processes and processes boundaries behaves quite differently from the original one in {\bfo}.\nb{CM: qui bisogna dire che questi sono tutti assiomi di bfo, non credo serva introdurre gli assiomi di bfo esplicitamente}
%
\bflist
\item[\dbth{pd_to_proc_or_pbnd}] $\thdolcedbmap \vdash \PDdcat(x) \ifif (\exists t(x \bfoiof{t} \procbcat) \lor \exists t(x \bfoiof{t} \pbndbcat))$

\item[\dbth{d2b_procdissect}] $\thdolcedbmap \nvdash x \bfoiof{t} \procbcat \land \bfoopart(x,y) \to \exists t'(y \bfoiof{t'} \procbcat)$
\hfill {\refbfoax{opart_proc1}}

\item[\dbth{d2b_opart_proc3}] $\thdolcedbmap \nvdash x \bfoiof{t} \procbcat \to \exists yt'(y \bfoiof{t'} \pbndbcat \land \bfoopart(y,x))$
\hfill {\refbfoax{opart_proc3}}

\item[\dbth{d2b_opart_pbnd2}] $\thdolcedbmap \nvdash x \bfoiof{t} \pbndbcat \land \bfoopart(y,x) \to \exists t'(y \bfoiof{t'} \pbndbcat)$
\hfill {\refbfoax{opart_pbnd2}}

\item[\dbth{d2b_pbndifif}] $\thdolcedbmap \nvdash \exists t(x\bfoiof{t}\pbndbcat) \ifif \exists y(\bfotpart(x,y) \land \exists t(y \bfoiof{t} \procbcat)) \land \exists t(\bfotregof(x,t) \land t \bfoiof{t} \tinstbcat)$
\mbox{} \hfill {\refbfoax{pbndifif}}

\item[\dbth{d2b_proctreg2tint}] $\thdolcedbmap \nvdash \exists t(x\bfoiof{t}\procbcat) \land \bfotregof(x,t) \to \exists t'(t' \bfoiof{t'}\tintbcat \land \bfotpart(t',t))$
\hfill {\refbfoax{proctreg2tint}}
\eflist


{\bfo} presupposes that processes and process boundaries have a different temporal nature, and constrains their relationship, which is quite complex, via a set of axioms. Vice versa \refdbdf{d2b_tinst} and \refdbdf{d2b_tint}---together with \refdbdf{d2b_proc} and \refdbdf{d2b_treg}---reduce this difference to atomicity and to the existence of bigger, with respect to temporal proper part, perdurants. Looking at the counterexamples used in the proof of the previous theorems, this different behavior about processes and process boundaries seems mainly due to the poor approximation of the notions of temporal instant and temporal interval provided by \refdbdf{d2b_tinst} and \refdbdf{d2b_tint} together with the low commitment of {\dolce} on the existence of perdurants.  For instance, in {\dolce} it is possible to have three temporally co-localized perdurants $p_1$, $p_2$, and $p_{12}$ all with atomic temporal locations and such that  the latter is the sum of the firsts, i.e., $\SUMd(p_{12}, p_1,p_2)$. There can also be two additional perdurants $p_3$ and $p_{123}$ such that the temporal location of $p_3$ does not overlap the one of $p_1$ and also $\SUMd(p_{123}, p_{12},p_3)$ holds. In this scenario, according to \refdbdf{d2b_proc} and \refdbdf{d2b_treg}, $p_{1}$ is a process because it has an atomic temporal location but it is not a temporal proper part of anything, $p_1$ is part of $p_{12}$, but $p_{12}$ is a process boundary  (against \refdbth{d2b_procdissect} and \refdbth{d2b_opart_pbnd2}) because it has an atomic temporal location and it is a proper temporal part of $p_{123}$. For another example, in {\dolce} it is possible to have perdurants that are not proper temporal part of any other perdurant and such that all their parts are temporally co-localized with them (possibly with an atomic temporal location) and in their turn are not proper temporal part of any other perdurant (e.g., discard $p_3$ and $p_{123}$ from the previous example). According to \refdbdf{d2b_proc} and \refdbdf{d2b_treg}, these perdurants as well as all their parts are classified as processes against \refdbth{d2b_opart_proc3}, \refdbth{d2b_pbndifif}, and \refdbth{d2b_proctreg2tint}. These examples may look `exotic', yet it is not easy to rule out them. One could get rid of some of these counterexamples by requiring that processes have a non-atomic temporal location. In this case, we need to remove \refdbth{pd_to_proc_or_pbnd} also because it is possible to have perdurants with an atomic temporal location that however are not temporal proper part of any other perdurant. 

The different `temporal behavior' of {\dolce} perdurants vs.~{\bfo} occurrents is highlighted also by \refdbth{stregof_opart_equiv}, which is relative to \refbfoax{stregof_opart_equiv_ax}. {\bfo} identifies\nb{CM: non sono sicuro su identifies} occurrent parthood between processes or process boundaries with spatiotemporal inclusion, it follows that different processes/process boundaries cannot be spatiotemporally co-localized. {\dolce} makes the opposite choice and allows for spatiotemporally co-localized perdurants. A similar situation arises between {\dolce} endurants  and {\bfo} continuants as highlighted by \refdbth{d2b_sregof_maten} and \refdbth{d2b_id_idcnt}. In {\dolce}, an amount of matter constituting a statue at a given time $t$ is different from the statue and it is not a $\TPd$-part of the statue at $t$ (and vice versa) even though the two endurants are spatially co-localized at $t$ (and possibly during their whole life), see \refdolceth{Kd_to_diff} and \refdolceax{Kd_colocalization}. Similarly, in {\dolce}, a sum of bricks is different from a wall even though they can mereologically coincide at a given time.
These differences are among the most important we found. In {\dolce}, space and time are not fundamental for the identity of entities. This gives room for a stratification of co-localized entities, which are further distinguished on the basis of their `modal' properties. In comparison, {\bfo} embraces a `reductionist' perspective by assuming that a given spatial or spatiotemporal region cannot be the location of different continuants or occurrents.\nb{CM: nel teorema \refdbth{stregof_opart_equiv} manca un riferimento al mapping che riguarda lo spazio tempo, da capire se vogliamo tenere questo o invece toglierlo}  
%
\bflist
\item[\bfoax{stregof_opart_equiv_ax}] $\exists t(x \bfoiof{t} \procbcat \lor x \bfoiof{t} \pbndbcat) \land \exists t(y \bfoiof{t} \procbcat \lor y \bfoiof{t} \pbndbcat) \to \\
\mbox{}\hfill{}(\bfoopart(x,y) \ifif \exists rs(\bfostregof(x,r) \land \bfostregof(y,s) \land \bfoopart(r,s)))$

\item[\dolceth{Kd_to_diff}] $\Kd(x,y,t) \to x \neq y$ 

\item[\dolceax{Kd_colocalization}] $\Kd(x,y,t) \to \forall s(\SLCd(x,s,t) \ifif \SLCd(y,s,t))$ 

\item[\dbth{stregof_opart_equiv}] $\thdolcedbmap \cup \{$\refdbdf{d2b_sregofocc}$\} \nvdash \exists t(x \bfoiof{t} \procbcat \lor x \bfoiof{t} \pbndbcat) \land \exists t(y \bfoiof{t} \procbcat \lor y \bfoiof{t} \pbndbcat) \to \\
\mbox{}\hfill{} \bfoopart(x,y) \ifif \forall t(\bfoexist(x,t) \to \exists rs(\bfosregofocc(x,r,t) \land \bfosregofocc(y,s,t) \land \bfocpart(r,s,t)))$

\item[\dbth{d2b_sregof_maten}]  $\thdolcedbmap \nvdash x \bfoiof{t} \mtenbcat \land y \bfoiof{t} \mtenbcat \land \bfosregof(x,r,t) \land \bfosregof(y,r,t) \to
\bfocpart(x,y,t) \land \bfocpart(y,x,t)$ 
\mbox{} \hfill{} \refbfoax{sregof_maten}

\item[\dbth{d2b_id_idcnt}] $\thdolcedbmap \nvdash \exists t(x \bfoiof{t} \idcntbcat \land y \bfoiof{t} \idcntbcat \land \neg(x \bfoiof{t} \objaggbcat) \land \neg(y \bfoiof{t} \objaggbcat) \land  \bfocpart(x,y,t) \land \bfocpart(y,x,t)) \to x=y$  \\ \mbox{} \hfill{} {\refbfoax{id_idcnt}}


\eflist


While endurants and occurrents, as well as $\TPd$ and $\bfocpart$, seem intuitively to correspond, \refdbth{d2b_id_idcnt} shows a relevant difference.\footnote{There are other differences: \refdbth{d2b_cpart_wext} shows that the $\bfocpart$ (as defined in \refdbdf{d2b_cpart}) does not satisfy the supplementation axiom, and \refdbth{d2b_ex_cprod} that the existence of the product is not guaranteed. Note that the existence of products cannot be inferred for $\bfoopart$ \refdbth{d2b_existence_oprod} either.}\nb{CM: forse portare la nota nel testo e aggiungere i teoremi/assiomi necessari} According to \refdbdf{d2b_cpart} and the definitions of the subclasses of $\cntbcat$, for independent continuants that are not spatial regions, $\bfocpart$ reduces to $\TPd$. However, for independent continuants that are not object aggregates, the original $\bfocpart$ is antisymmetric, but is not so the relation defined by \refdbdf{d2b_cpart}, that in this case it is equivalent to $\TPd$. As we have seen, this fact has an important impact on the endurant/continuant entities accepted by the two theories. One could recover the antisymmetry by changing the definition of $\bfocpart$. This amounts to break the correspondence between $\TPd$ and $\bfocpart$ for independent continuants by `injecting' antisymmetry into \refdbdf{d2b_cpart}, e.g.,
\bflist
\item[--] $\bfocpart(x,y,t) \sdef x \bfoiof{t} \cntbcat \land y \bfoiof{t} \cntbcat \land ($\parbox[t]{\textwidth} {$(\TPd(x,y,t) \land (\neg \TPd(y,x,t) \lor x=y)) \lor \Pd(x,y) \lor \\ 
\exists zu(\DQTd(x,z) \land \DQTd(y,u) \land \TPd(z,u,t)))$}
\eflist
%
This new mapping has the advantage of $(i)$ recovering the antisymmetry as in the original {\bfo}; and of $(ii)$ making explicit one difference between $\bfocpart$ and  $\TPd$. However, we can have endurants that in {\dolce} are linked by $\TPd$ that are still imported under $\idcntbcat$ but are not linked by the newly defined $\bfocpart$. For instance, in the previous example, the (sum of the) bricks and the wall would not be $\bfocpart$-related. In these cases \refdbth{d2b_sregof_maten} still holds. %  \refbfoax{sregof_maten} still fails. 
To solve this problem one could explicitly specialize the definition of $\bfocpart$ to take into account these cases, e.g.,  
\bflist
\item[--] $\bfocpart(x,y,t) \sdef $\parbox[t]{\textwidth} {$x \bfoiof{t} \cntbcat \land y \bfoiof{t} \cntbcat \land \\
($\parbox[t]{\textwidth} {$(\TPd(x,y,t) \land (\neg \TPd(y,x,t) \lor x=y)) \lor \\ 
\exists sr(\SLCd(x,s,t) \land \SLCd(y,r,t) \land \Pd(s,r)) \lor \\ 
\Pd(x,y) \lor \exists zu(\DQTd(x,z) \land \DQTd(y,u) \land \TPd(z,u,t)))$}}
\eflist
However, in this way the antisymmetry of $\bfocpart$ is lost once more because the bricks and the wall would $\bfocpart$-coincide at $t$ even though they are different. One could then change the definition of $\bfosregof$ stating that the bricks (or the wall) have not spatial location at $t$, or that at $t$ they have different spatial locations. Similarly in the case of the statue and the clay. However, this option seems too strong to pursue.\footnote{We have a similar problem with the class of occurrents as highlighted by \refdbth{stregof_opart_equiv}.}

\medskip
These examples suggest to follow a different approach possibly more respectful of the original commitments of the two ontologies. Let us go back to the classical example of the statue and the clay, or to the one of the brick and the wall. There are two different endurants which are spatially coincident at least at a given time. One can think that \refbfoax{id_idcnt} and \refbfoax{sregof_maten}\nb{CM: forse si può fare riferimento agli ai teoremi o vice versa si possono introdurre gli assiomi e semplicemente dire che questi non valgono in dolce+def} aim at ruling out these kinds of examples: it is not a matter of avoiding spatial coincidence or parthood between the brick and the wall, their goal, one could argue, is to rule out one of these two entities, i.e., to rule out from the domain  of quantification, either the brick or the wall.\footnote{Clearly, this is in contrast with (M2) and with the mappings that, vice versa, allow to import all the endurants of {\dolce} into {\bfo}.} One can then adopt a strategy that `filters out' some of the {\dolce} endurants. Setting this filter is not trivial. For the case of the statue and the clay, one can rely on the fact that constitution is a sort of order relation in {\dolce}. One can maintain the substratum (the clay) in the domain and discard the statue due to the fact that it is the substratum that constitutes the statue, not vice versa. However, in the case of the bricks and the wall we cannot rely on $\TPd$ because the sum of the bricks is part of the wall and the wall is part of the sum of the bricks. One needs to find a different relation or some other principle to rule out one of these. But this is not all. Let us assume that we do not import the statue into {\bfo}. Does it mean that {\bfo} cannot talk about statues or that it relies on a different representation? After all, it classifies the clay under amount of matter as well as under statue. One would need to translate the {\dolce} claims about the statue into {\bfo} claims about the amount of clay %that (in {\dolce}) constitutes the statue 
(since it is the only element left in the domain of {\bfo}). The consequences of this approach deserve more investigation. It is however important to note that this first analytical step allows us to understand the differences and to individuate the main problems that a mapping needs to address.

The analysis of more refined correspondences, like \refdbth{MorPOB_not_cnt}, \refdbth{inst_MorPOB_to_mten}, and \refdbth{mten_not_PED}, shows that temporally extended amounts of matters/physical objects are material entities. Yet, the other direction does not hold: non-physical endurants could be very well be material entities. However,  \refdbth{pd_to_proc_or_pbnd}, \refdbth{MorPOB_not_cnt}, together with \refdbth{F_not_cnt}, \refdbth{Q_not_cnt}, and \refdbth{NPED_not_cnt}, show that there are endurants and qualities, all of which are particulars (see \refdbth{-AB_particulars}), that are neither continuants nor occurrents. This means that the partition of particulars into continuants and occurrents assumed by {\bfo} is not preserved by the mapping, hence, some endurants and qualities  are in a sort of ontological `limbo'. 
%
\bflist
\item[\dbth{MorPOB_not_cnt}] $\thdolcedbmap \nvdash (\Mdcat(x) \lor \POBdcat(x)) \to \exists t(x \bfoiof{t} \cntbcat)$

\item[\dbth{inst_MorPOB_to_mten}] $\thdolcedbmap \vdash (\Mdcat(x) \lor \POBdcat(x)) \land \exists t(\TLCd(x,t) \land \neg \ATd(t)) \to \exists t(x \bfoiof{t} \mtenbcat)$

\item[\dbth{mten_not_PED}] $\thdolcedbmap \nvdash \exists t(x \bfoiof{t} \mtenbcat) \to \PEDdcat(x)$

\item[\dbth{F_not_cnt}] $\thdolcedbmap \nvdash \Fdcat(x) \to \exists t(x \bfoiof{t} \cntbcat)$

\item[\dbth{Q_not_cnt}] $\thdolcedbmap \nvdash \Qdcat(x) \to \exists t(x \bfoiof{t} \cntbcat)$

\item[\dbth{NPED_not_cnt}] $\thdolcedbmap \nvdash \NPEDdcat(x) \to \exists t(x \bfoiof{t} \cntbcat)$
\eflist

On the other hand, the imported entities do not necessarily cover all the categories of {\bfo}. There are at least two reasons for that, one factual and one structural. $(i)$ The factual reason is that some of the {\dolce} categories may be empty. For instance, if there are no features, then $\imenbcat$ would be empty. Consequently, \refbfoax{univ_are_insta}, which states that all universals are non-empty, is not preserved. $(ii)$ The structural reason is that, according to the mappings, some categories (as defined in the mappings) are necessarily empty. %\nb{SB: ho cambiato un po' il paragrafo e in particolare aggiunto qui 'new' categories, check}
  This is the case of universals (when there is at least a time in the domain of quantification, see \refdbth{all_particulars}). One solution is to `force' some types of entities in the ontological `limbo' to be mapped into {\bfo} universals. We already discussed the import of some regions into universals, similarly one could assume that specific subclasses of non-physical endurant ($\NPEDdcat$) are concepts behaving like {\bfo} universals. This approach has several consequences on the set of theorems.\nb{CM: ci sarebbero un sacco di altre osservazioni da fare sui teoremi che non seguono%, ma questo richiederebbe di entrare in più dettagli tecnici ancora e di come minimo raddoppiare la sez., non se se ritenete che le considerazioni sopra siano sufficienti
\\
SB: lascierei così almeno per ora}


%\medskip
One way to choose among the different mapping approaches is to look at what the mapping optimize. For instance, strict and restrictive mappings, carefully set to not allow exchanges of entities which would be in the ontological `limbo' of the target ontology, might be preferred in the context of data transfer, they are safer in this context. Mappings focusing on ontological significance and the maximization of the number of imported entities are instead arguably better for the goals of comparing ontologies, highlighting core differences, and establishing general interoperability results.

%=====================================
\section{The mapping from {\bfo} to {\dolce}}\label{sect_b2d}
%=====================================

%======================================
\subsection{Mappings}\label{sect_mappings_b2d}
%======================================

In this section we introduce syntactic definitions of {\dolce} notions in terms of {\bfo} primitives. Analogously to the {\dolce} to {\bfo} direction of Sect. \ref{sect_d2b} and with similar motivations (see the analysis of Sect. \ref{sect_analysis_b2d}), we present only a partial mapping covering a subset of the categories and primitives of {\dolce}.  

As discussed in Sect.~\ref{sect_diff_resolution}, in {\dolce}, the agentive and social dimensions play an important role in characterizing the subcategories of physical object ($\POBdcat$) and non-physical endurant ($\NPEDdcat$) but those dimensions are not considered in {\bfo}. Similarly, the subcategories of perdurant ($\PDdcat$) rely on the notions of homeomericity and cumulativity, but these notions are beyond the scope of {\bfo}. We then rule out all these subcategories from the mappings. Furthermore, while it is not clear how the distinction between amounts of matter and physical objects can be made, for features we can rely on the {\bfo} categories of site, continuant fiat boundary and fiat object. Thus, within the category of physical endurants ($\PEDdcat$) we consider only the category of features ($\Fdcat$). 

We will see that physical endurants correspond to independent continuants which are not regions,
see \refbddf{b2d_PEDdcat}, while the only entities that can be mapped to non-physical endurants are generically dependent continuants, see \refbddf{b2d_NPEDdcat}. In {\dolce}, arbitrary sums ($\ASdcat$) have necessarily a physical and a non-physical part, see \refdolceax{TPd_AS}. To match this axiom, in {\bfo} we should find continuants which are not spatial regions and are neither independent nor generically dependent continuants. The only option is that of specifically dependent continuants. However, following our guidelines in Sect. \ref{sect_methodology}, these are mapped to {\dolce} physical qualities as stated by \refbddf{b2d_PQdcat}. As a result, $\ASdcat$ would be necessarily empty. For this reason, this category is not covered by the mapping.   
%
\bflist
\item[\dolceax{TPd_AS}] $\ASdcat(x) \to \exists yzut(\TPd(y,x,t) \land \PEDdcat(y) \land \TPd(z,x,u) \land \NPEDdcat(z))$  
\eflist

In Sect.~\ref{sect_analysis_d2b} we have seen that one of the main differences between {\dolce} and {\bfo} concerns the reduction of the parthood relations to spatial and spatiotemporal inclusions. In particular, in {\bfo} there are no spatially co-localized distinct independent continuants (that are not object aggregates) and there are no spatiotemporally co-localized distinct occurrents. We then lack one of the main basis to build a constitution relation among endurants and perdurants. In principle, one could see the relation between a generic dependent continuant and the mereological sum of its carriers at a given time as a form of constitution. This solution is debatable and would model only a very limited notion of constitution, quite different from the general one in {\dolce}. Thus, we refrain from introducing $\Kd$ in the mappings.  

Concerning qualities, {\bfo} accepts only qualities inhering in independent continuants. From the point of view of {\dolce} (considering \refbddf{b2d_PEDdcat} below), this restricts the mapping to physical qualities only. 

In the case of regions, we have just time intervals and space regions but to locate entities in time and space we need to rely only on $\TLCd$ and $\SLCd$ because temporal and spatial locations have no correspondent entities in {\bfo} and the $\QLd$ and $\TQLd$ relations do not make sense in this setting. Alternatively, as previously discussed, one could import some {\bfo} universals into {\dolce} regions and assume that instance-of corresponds to ($\DQTd$ composed with) $\TQLd$. In this cases, the ISA relation corresponds to the primitive $\Pd$ (that in {\dolce} is defined on regions). Beside the mapping presented in this section, this other alternative is also worth development. We leave it for future work. 
%

Finally, the categories of \emph{fact} and \emph{set} that in the original taxonomy of {\dolce} appear under abstract ($\ABdcat$) have not been explicitly taken into account in {\dolce}-\textsc{cl}, we safely ignore them here. 
 

\smallskip
Summing up, among the {\dolce} notions, in the following we introduce: 
\begin{enumerate}[$(i)$]
\item syntactic definitions for the primitive relations: $\Pd$, $\TPd$, $\TLCd$, $\SLCd$, $\PCd$, $\DQTd$, $\EXDd$; and 
\item syntactic definitions for the categories: $\EDdcat$, $\PDdcat$, $\Qdcat$, $\ABdcat$, $\Fdcat$, $\PEDdcat$, $\NPEDdcat$, $\PQdcat$, $\Rdcat$, $\TRdcat$, $\PRdcat$, $\Tdcat$, and $\Sdcat$.
\end{enumerate}

Below we list these syntactic definitions together with a short informal description. A deeper analysis about these definitions and their impact on the preservation of the axioms of {\dolce} is done in Sect.~\ref{sect_analysis_b2d}.
%
\bflist
\item[\bddf{b2d_PDdcat}] $\PDdcat(x)\sdef x \bfoiof{} \procbcat \lor x \bfoiof{} \pbndbcat$

\vspace{1pt}
Perdurants coincide with the disjunction of processes and process boundaries. 

\item[\bddf{b2d_EDdcat}] $\EDdcat(x)\sdef x \bfoiof{} \cntbcat \land \neg(x \bfoiof{} \sregbcat) \land \neg(x \bfoiof{} \sdcntbcat)$

\vspace{1pt}
We rule out from endurants temporal and spatial regions that are mapped, respectively, to time intervals and space regions, see \refbddf{b2d_Tdcat} and \refbddf{b2d_Sdcat}.

\item[\bddf{b2d_PEDdcat}] $\PEDdcat(x)\sdef x \bfoiof{} \idcntbcat \land \neg(x \bfoiof{} \sregbcat)$

\vspace{1pt}
Physical endurants are independent continuants that are not spatial regions.

\item[\bddf{b2d_Fdcat}] $\Fdcat(x)\sdef x \bfoiof{} \sitebcat \lor x \bfoiof{} \cfbndbcat \lor x \bfoiof{} \fobjbcat$

\vspace{1pt}
Features are the union of sites, continuant fiat boundaries, and fiat objects.

\item[\bddf{b2d_NPEDdcat}] $\NPEDdcat(x)\sdef x \bfoiof{} \gdcntbcat$

\vspace{1pt}
Non-physical endurants coincide with generically dependent continuants.

%{\color{red} ****the class of arbitrary sum is necessarily empty because continuants that are not specifically dependent are partitioned in idcnt e gdcnt and sreg are mapped to regions, però devo cedo mettere un mapping del genere $\ASdcat(x) \sdef \EDdcat(x) \land \neg \PEDdcat(x) \land \neg \NPEDdcat(x)$}

\item[\bddf{b2d_PQdcat}] $\PQdcat(x)\sdef x \bfoiof{} \sdcntbcat$

\vspace{1pt}
Physical qualities coincide with specifically dependent continuants. Note that relational qualities and realizable entities like roles and dispositions are imported into physical qualities.\nb{CM: vedi nota importante tolta} 

%{\color{red} ****however we do not have spatial locations, i.e., here we are defining only physical qualities that are not spatial locations... do we need to add something like $\SLdcat(x) \sdef \bfopartic(x) \land \neg \bfopartic(x)$ ?}

%{\color{blue} old definition:
%
%$\PQdcat(x)\sdef x \bfoiof{} \sdcntbcat \land \neg(x \bfoiof{} \rqltbcat)$
%
%****actually {\dolce} seems compatible with relational qualities, it is enough to distinguish $\DQTd$ from other kinds of specifically dependencies and add a double $\DQTd$/$\SDd$ dependence for these qualities}  

%{\color{red} ****this mapping is maybe too `weak', it includes into {\dolce} qualities also {\bfo}-roles and {\bfo}-dispositions that, from the examples in the documentation, I'm not sure are {\dolce} qualities;
%however, in the new version of {\dolce} the temporal extension of qualities is included in, but possibly different from, the temporal extension their bearers (in any case I don't see an axiom guaranteeing the temporal co-extensionality of {\bfo} qualities too)}
%
%{\color{red} ****more generally, one can try to understand $(i)$ if roles can be mapped to  \emph{qua-entities}; and $(ii)$ if dispositions can be organized in a space (following what done for capabilities by Stefano and Laure) one could have a single disposition-quality changing its location in this space}
%
%{\color{red} ****{\bfo} does not have temporal or abstract qualities}

\item[\bddf{b2d_Qdcat}] $\Qdcat(x) \sdef \PQdcat(x)$ 

\vspace{1pt}
Only physical qualities exist ({\bfo} has only qualities of independent continuants)

\item[\bddf{b2d_Tdcat}] $\Tdcat(x)\sdef x \bfoiof{} \tregbcat$

\vspace{1pt}
Time intervals coincide with temporal regions. Here one could consider a stronger mapping, i.e., assume that $\Tdcat$ corresponds to $\tintbcat$. However, following (M2) we try here to import all the spatial regions.

\item[\bddf{b2d_TRdcat}] $\TRdcat(x) \sdef \Tdcat(x)$

\vspace{1pt}
Among temporal regions there are only time intervals.

\item[\bddf{b2d_Sdcat}] $\Sdcat(x)\sdef x \bfoiof{} \sregbcat$

\vspace{1pt}
Space regions coincide with spatial regions.

\item[\bddf{b2d_PRdcat}] $\PRdcat(x)\sdef \Sdcat(x)$

\vspace{1pt}
Among physical regions there are only space regions.

%\vspace{1pt}
%{\color{blue} {old definition}: 
%
%$\PRdcat(x)\sdef \Sdcat(x) \lor \exists y(\Qdcat(y) \land y \bfoiof{} x)$
%
% ****this old definition ATTEMPT to include universals classifying {\bfo} sdcnt-entitites under physical region; in this way instantiation could represent the TQL and the isa relation could approximate parthood (actually parthood is intensional while isa is extensional)}

\item[\bddf{b2d_Rdcat}] $\Rdcat(x) \sdef \TRdcat(x) \lor \PRdcat(x)$

\item[\bddf{b2d_ABdcat}] $\ABdcat(x)\sdef \Rdcat(x)$

\vspace{1pt}
Among abstracts there are only time intervals and space regions.


%\item[\bddf{b2d_PTdcat}] $\PTdcat(x) \sdef \EDdcat(x) \lor \PDdcat(x) \lor \Qdcat(x) \lor \ABdcat(x)$

\item[\bddf{b2d_TLCd}] $\TLCd(x,t)\sdef (\PDdcat(x) \lor \EDdcat(x) \lor \Qdcat(x)) \land \Tdcat(t) \land \forall u(\bfoexist(x,u) \ifif \bfotpart(u,t))$

\vspace{1pt}
The temporal location of $x$ is the maximal time at which $x$ exists. In {\bfo} $\bfotregof$ is defined only on processes and process boundaries). 

%{\color{blue} ****added $\Tdcat(t)$}

\item[\bddf{b2d_Pd}] $\Pd(x,y)\sdef (((\PDdcat(x) \land \PDdcat(y)) \lor (\Tdcat(x) \land \Tdcat(y))) \land \bfoopart(x,y)) \lor (\Sdcat(x) \land \Sdcat(y) \land \forall t(\bfoexist(x,t) \to \bfocpart(x,y,t)))$

\vspace{1pt}
For perdurant and time intervals $\Pd$ coincides with $\bfoopart$ while for space regions it reduces to constant parthood, i.e., $x$ is a temporary part of $y$ during its whole existence (note that in {\bfo}, all the spatial regions exist at least at a time). 

%\vspace{1pt}
%{\color{blue} {old definition}: 
%
%$\Pd(x,y)\sdef \ $\parbox[t]{\textwidth} {$(((\PDdcat(x) \land \PDdcat(y)) \lor (\Tdcat(x) \land \Tdcat(y))) \land \bfoopart(x,y)) \lor \\ (\Sdcat(x) \land \Sdcat(y) \land \exists t(\bfocpart(x,y,t)))  \lor \\
% (\PRdcat(x) \land \PRdcat(y) \land \neg\Sdcat(x) \land \neg\Sdcat(y) \land \bfoisa(x,y))$}
%
%****the extensionality of $\Pd$ probably is not provable in general because we don't have this kind of property for $\bfoisa$}

\item[\bddf{b2d_TPd}] $\TPd(x,y,t) \sdef \EDdcat(x) \land \EDdcat(y) \land \bfocpart(x,y,t)$

\vspace{1pt}
For endurants $\TPd$ coincides with $\bfocpart$. 

\item[\bddf{b2d_PCd}] $\PCd(x,y,t) \sdef \EDdcat(x) \land \PDdcat(y) \land \exists z(\bfotpart(y,z) \land \bfoparticin(x,z,t))$

\vspace{1pt}
$\bfoparticin$ is not defined on process boundaries while {\dolce} does not impose any constraint on the kind of the process involved in $\PCd$. The existential quantification on $z$ aims to mitigate this difference. 


\item[\bddf{b2d_DQTd}] $\DQTd(x,y) \sdef \bfoinh(x,y)$

%{\color{red} ****this seems too restrictive because in {\bfo}  specifically dependent continuants specifically depend only on independent continuant \refbfoax{sdcnt_def} while in {\dolce} qualities can inhere also in dependent endurants and in perdurants; we are `interpreting' $\DQTd$ in a restrictive way, also in this case we would have $\bfoinh(x,y) \to \DQTd(x,y)$ but not vice versa (again, see discussion done for \refbddf{b2d_EDdcat}, \refbddf{b2d_PEDdcat}, and \refbddf{b2d_NPEDdcat})}

\item[\bddf{b2d_EXDd}] $\EXDd(x,y,t) \sdef (\bfosdep(x,y) \land \bfoexist(x,t)) \lor \bfogdep(x,y,t) \lor \bfoparticin(x,y,t) \lor \bfoparticin(y,x,t)$


\vspace{1pt}
$\EXDd$ is the generalization of $\bfosdep$, $\bfogdep$, and $\bfoparticin$ ($\bfoparticin$ is a sort of mutual dependence).

%{\color{red} ****this definition is bad; $\EXDd$ is strictly stronger than $\bfosdep$/$\bfogdep$ because  it can hold $(i)$ only at some times (differently from $\bfosdep$) and $(ii)$ between specifically/generically dependent endurants (according to \refbddf{b2d_Qdcat} specifically dependent continuants are mapped to qualities; also in this case it seems more an inclusion (right to left direction) than a definition (but see the discussion relative to \refbddf{b2d_EDdcat}, \refbddf{b2d_PEDdcat}, and \refbddf{b2d_NPEDdcat})}

\item[\bddf{b2d_SLCd}] $\SLCd(x,s,t) \sdef $\parbox[t]{\textwidth} {$(x \bfoiof{t} \idcntbcat \land \neg(x \bfoiof{t} \sregbcat) \land \bfosregof(x,s,t)) \lor \\ 
(x \bfoiof{t} \gdcntbcat \land \exists y(\bfogdep(x,y,t) \land \forall z(\bfogdep(x,z,t) \to \bfocpart(z,y,t)) \land \bfosregof(y,s,t))) \lor  \\  
(\Qdcat(x) \land \exists y(\bfoinh(x,y) \land \forall z(\bfoinh(x,z) \to \bfocpart(z,y,t)) \land \bfosregof(y,s,t))) \lor  \\  
(\PDdcat(x) \land \exists y(\bfostregof(x,y) \land \bfosproj(y,s,t)))$}

\vspace{6pt}
For independent continuants that are not spatial regions, $\SLCd$ coincides with $\bfosregof$. The spatial location, at $t$, of a generically dependent continuant $x$ is the spatial region of the maximal entity (if it exist) on which $x$ generically depends on at $t$.  The spatial location of a quality $x$ at $t$ is the spatial region of the maximal (at $t$) entity (if it exist) in which $x$ inheres. The spatial location of a perdurant is the spatial projection of its spatiotemporal location.  
(Note that in {\bfo}, no axiom guarantees that a specifically dependent continuant $x$ inheres in a unique continuant, therefore for qualities we define the spatial location only when there is a `maximal continuant' in which $x$ inheres in. Similarly for generic dependent continuants.)

%%\vspace{3pt}
%%
%%{\color{red} ****in {\dolce} spatial location applies to endurants, qualities, and perdurants.}
%%
%%{\color{red} ****in {\bfo}, no axiom guarantees that a specifically dependent continuant $x$ inheres in a unique continuant, therefore for qualities we define the spatial location only when there is a `maximal continuant' in which $x$ inheres in; similarly for generic dependent continuants (alternatively on could rely on the sum of the spatial regions occupied by the continuants $x$ inheres in or depends on, but {\bfo} does not commit on the existence of sums).}
%
%\item[--] $\Kd(x,y,t) \sdef \ ????$
%
%{\color{red} ****I don't think this can be defined because {\bfo} does not allow different endurants to be spatially co-localized (see also the example changing a leg in the paper of Neil and Alan in the special issues on top-levels)}
%
%{\color{red} ****this also make explicit that some classes of {\dolce} are necessary empty, namely all the classes of entities that depends on other entities spatially co-localized. One could then try to build these entities in a complex way: (1) reify the instantiation of given universals, e.g., one can reify the fact that an amount of matter $m$ is a statue at a given time $m \bfoiof{t} \cn{statue}$; (2) collect all the reified states of the same kind (e.g., statue-states) that are in given relations defined in terms of (spatio-temporal) relations between amount of matters, or given kinds of transformation processes that apply on the amount of matter. This is interesting but complex, it would be better done at the level of models in order to avoid to modify the domain of quantification of {\bfo}.}
%
%\item[--] $\QLd(x,y) \sdef \ ????$
%
%{\color{red} ****{\bfo} qualities are defined only for continuants (and in any case qualities  do not have qualia, see discussion in \refbddf{b2d_TQLd})}
%
%\item[--] $\TQLd(x,y,t) \sdef \ ????$
%
%{\color{blue} old definition:
%
%$\TQLd(x,y,t) \sdef \Qdcat(y) \land y \bfoiof{t} x \land \neg \exists z(y \bfoiof{t} z \land \bfoisa(z,x) \land \neg \bfoisa(x,z))$
%
%****the only ``abstract entities'' considered by {\bfo} are temporal and spatial regions (also spatiotemporal regions that however are not considered by {\dolce})
%
%****in general one could try a link with universals, i.e., one could assume that the instances of $\qltbcat$ are partitioned by means of a set of universals (identified by a predicate $U$) each one corresponding to a quality space in {\dolce}, and then assume that the quale of a quality $q$ at $t$ is the more refined universal under one $U$-universal of which $q$ is an instance of at $t$, i.e.,  $\TQLd(x,y,t) \sdef U(x) \land y \bfoiof{t} x \land \neg \exists u(y \bfoiof{t} u \land u \neq x \land \forall z(z \bfoiof{t} u \to z \bfoiof{t} x))$}
\eflist

One could think that these mappings are too restrictive: in {\dolce} physical endurants do not include only independent continuants, non-physical endurants do not include only generically dependent continuants (a quite special kind of entities), etc. This may suggest to include  $x \bfoiof{} \idcntbcat \land \neg(x \bfoiof{} \sregbcat) \to \PEDdcat(x)$ as well as $x \bfoiof{} \gdcntbcat \to \NPEDdcat(x)$, but not the converse formulas. We need however to remember that here we are considering the mapping from {\bfo} to {\dolce}. The starting point is the domain of {\bfo}, i.e., we need first of all to try to classify the entities \emph{in the domain of {\bfo}} in terms of the categories of {\dolce}. By looking at the mappings it is easy to see that all the particulars\footnote{We already discussed the possibility to import universals as regions or non-physical endurants, see Sect. \ref{sect_analysis_d2b}.} of {\bfo} are classified in terms of {\dolce} categories except spatiotemporal regions:\footnote{We already discussed the fact that spatiotemporal regions seem superfluous, see Sect. \ref{sect_mappings_d2b}. However, it is easy to modify the mappings to import spatiotemporal regions under {\dolce} regions.} $(i)$ independent continuants which are not spatial regions are mapped to physical endurants; $(ii)$ spatial regions to space regions; $(iii)$ generically dependent continuants to non-physical endurants; $(iv)$ specifically dependent continuants to {\dolce} qualities; $(v)$ process and process boundaries to perdurants; and $(vi)$ temporal regions to time intervals. The fact that {\dolce} intuitively accepts, for instance, additional physical or non-physical endurants shows that, in general, {\dolce} has a domain larger than the domain of {\bfo}. Our methodological choice (M2) does not allow the mappings to `enrich' the domain of {\bfo} with new entities.  However, by introducing the mapping at the semantic level, i.e., in this case, as an operator translating {\bfo}-structures into a {\dolce}-structure, one could enrich the original domain of {\bfo} by set-theoretically building new entities. For instance, this is the strategy followed in the literature to map theories of time based on points to theories based on intervals.\footnote{See van Benthem, J., \emph{The Logic of Time}, Springer, 2nd ed., 1991.} The theory based on intervals does not admit points, but points can be built as set of intervals staying in a given relation that is definable in the theory of intervals.  Analogously, one can think of building some entities, e.g., statues as opposed to amount of clay, starting from the amounts of clay and the specific kinds of universals they instantiate. The investigation of `semantic mappings' is not discussed here. 

%{\color{red} ****The last three definitions (PED, NPED, F) nicely illustrate the problem of mappings when one has ontologies with different domains. These definitions approximate $\EDdcat$, $\PEDdcat$, and $\NPEDdcat$ from the point of view of {\bfo}, i.e., they individuate \emph{among the entities in the domain of {\bfo}} the ones that reasonably match the idea of $\EDdcat$, $\PEDdcat$, and $\NPEDdcat$ in {\dolce}. However, concerning these kinds of entities, {\dolce} seems to have a larger domain, i.e., on would want to say, for instance for $\NPEDdcat$s, just $x \bfoiof{} \gdcntbcat \to \NPEDdcat(x)$ while in general one could have other $\NPEDdcat$s. This however is true only by considering the domain of {\dolce} while in the domain of {\bfo}, generically dependent continuant cover the whole possible $\NPEDdcat$s, there are no additional entities in {\bfo} that can be seen as $\NPEDdcat$s. In terms of semantic links between structures, the set of mappings can be seen as a sort of `operator' that (possibly) translates a {\bfo}-structure into a {\dolce}-structure, but one always needs to start from the entities in the domain of {\bfo} and to construct a {\dolce}-structure in terms of these entities. One could introduce in the {\dolce}-structure some entities that are set-theoretical constructions, but the basic elements of these set-theoretical constructions always are in the {\bfo}-domain. This is different from the perspective where we consider a sort of `union' of the two theories (that are assumed to have disjoint vocabularies) and where we have a common domain that allows us to determine set-theoretical relations between the primitives classes and relations.}
%
%{\color{red} ****{\dolce} features contains other kinds of entities, it is only an inclusion, but see the discussion relative to \refbddf{b2d_EDdcat}, \refbddf{b2d_PEDdcat}, and \refbddf{b2d_NPEDdcat}}

%=================================
\subsection{Check of the preservation of the original {\dolce} axioms}\label{sect_check_dolce_preservation}
%=================================

As discussed in Sect.~\ref{sect_diff_resolution} (and made explicit in Sect.~\ref{sect_mappings_b2d}) some {\dolce}-axioms involve relations or categories that have not been defined in the mappings. In what follows, when possible and relevant we consider approximations of these axioms as expressible in the {\bfo} language, otherwise we set them aside.
%We do not consider these axioms or, when possible, only approximations.

Furthermore \refbfoax{univ_are_insta} states that all the universals are non-empty. From an applicative perspective, this is a strong constraint because it forces the user to model even entities which might not be relevant to the application (and perhaps time consuming to analyze). Note that, once \refbfoax{univ_are_insta} is removed, the existential constraints in $\thbfo$ are compatible with the instantiation of only some categories (see the counterexample used below). From a technical perspective, \refbfoax{univ_are_insta} requires very large models that make the identification and management of counterexamples very difficult even using dedicated software. For these reasons, in this section we do not consider \refbfoax{univ_are_insta}.   

%Let us include \refdbdf{d2b_mten}-\refdbdf{d2b_gdep} into $\dbmap$.
In the following, $\thbfobdmap$ indicates the `extension' of $\thbfo \setminus \{$\refbfoax{univ_are_insta}$\}$ with the mappings introduced in Sect.~\ref{sect_mappings_b2d} together with the syntactic definitions introduced in Sect.~\ref{sect_dolce}, i.e., $\thbfobdmap = \{\thbfo \setminus \{$\refbfoax{univ_are_insta}$\}\} \cup \{$\refdbdf{b2d_PDdcat}-\refdbdf{b2d_SLCd}$\} \cup \{$\refdolcedf{def_PPd}-\refdolcedf{dfSDd}$\}$.
%
%In the following, $\thbfo$ is the set of axioms characterizing {\bfo} as presented in Section \ref{sect_bfo},  $\bdmap$ is the set of mappings from {\bfo} to {\dolce}, i.e., $\dbmap = \{\refdbdf{b2d_PDdcat}$-$\refdbdf{b2d_SLCd}\}$, and $\thbfobdmap =  \thbfo \cup \dbmap$.
%
%Some of the axioms in {\dolce} involve several subcategories and/or relations of {\dolce} that has not been defined in terms of {\bfo}. In the following we do not consider these axioms. 

%For this direction of the mappings, we still lack the proofs or the counterexamples for some axioms (expressible by the limited vocabulary taken into account in the mappings) among the ones reported in Sect.~\ref{sect_dolce}. We label these formulas in red by prefixing an asterisk. In some cases, we provide our expectation (just an expert guessing at this stage) followed by a short motivation.

 Several proofs has been verified by using theorems provers, they are reported in Sect.~\ref{bd_mapping}.
 
 %======================================
\subsection{Analysis}\label{sect_analysis_b2d}
%======================================

The results discussed in Sect.~\ref{d2b_taxonomy} show that, with the exception of the spatiotemporal regions, all the particulars of {\bfo} find their place in {\dolce}. The possibility to avoid spatiotemporal regions, once one has the temporal and the spatial locations of entities, has been analyzed in Sect.~\ref{occupies_streg}. For importing {\bfo} universals into {\dolce} we already suggested some possibilities, e.g. mapping universals to regions or to non-physical endurants. Both alternatives would require to modify the mappings \refbddf{b2d_NPEDdcat} and/or  \refbddf{b2d_Rdcat} (and eventually \refbddf{b2d_TRdcat} and \refbddf{b2d_PRdcat}). Vice versa, some of the original categories of {\dolce} remain necessarily `empty'. It is the case of $\ASdcat$, $\ARdcat$, $\TQdcat$, and $\AQdcat$. Other categories are drastically reduced: physical qualities/regions reduce to spatial locations/space regions and temporal regions reduce to time intervals. It seems that {\dolce} accepts a larger variety of entities. This is also confirmed by the possibility to have in {\dolce} spatially co-localized, and possibly layered, entities linked by coincidence (mutual $\TPd$-parthood at a time) and constitution. Whether this lareger variety of entities translates into an higher expressive power needs to be investigated. As shown in the case of spatiotemporal regions, more complex mappings could at least mitigate the gap.

Let us now take into account the results in Sect.~\ref{sect_check_dolce_preservation}. Even though the proofs of some theorems are still lacking, some observations can already be made. It results quite evident that the majority of the original axioms of {\dolce} are not preserved by the mappings. At first sight, one could take this as an evidence of the bad quality of mappings, or of a genuine difference between the two ontologies, or conclude, more drastically, that a comparison of the mappings in the two directions shows that {\bfo} is `weaker' than {\dolce}. A deeper analysis of the reasons underlying the counterexamples of some theorems may suggest a more complex situation with pros and cons on both sides.  

Few, and quite technical, differences between the two theories seem to cause several systematic problems for a comprehensive mapping. In {\dolce}, $\PREd$ is defined only for the entities that have a temporal extension, see \refdolcedf{dfPREd}, while in {\bfo} $\bfoexist$ is a primitive relation.\footnote{One could think that $\bfotregof$ corresponds to $\TLCd$ but actually $\bfotregof$ is defined only for occurents and it is only quite minimally linked to $\bfoexist$.}  The counterexample for \refbdth{b2d_d_partictime} shows that in {\bfo} an entity may exist at two non overlapping temporal regions without existing at any bigger temporal region. In these situations the condition $\forall u(\bfoexist(x,u) \ifif \bfotpart(u,t))$ in the definition of $\TLCd(x,t)$, see \refbddf{b2d_TLCd}, is never satisfied and the relation $\PREd$ is always `empty'. This problem has a huge impact in the proofs in Sect.~\ref{sect_check_dolce_preservation} because $\PREd$ is heavily used in the characterization of the primitives of {\dolce}. At the same time, it is a quite subtle problem whose origin seems to have a technical, more than ontological, nature; at the end,  in both ontologies the entity exists at both times. 
%
\bflist
\item[\dolcedf{dfPREd}] $\PREd(x,t) \sdef \exists u(\TLCd(x,u) \land \Pd(t,u))$ 

\item[\bdth{b2d_d_partictime}] $\thbfobdmap \nvdash (\EDdcat(x) \lor \PDdcat(x) \lor \Qdcat(x)) \to \exists t(\TLCd(x,t))$ 
\hfill{\refdolceax{d_partictime}} 

\item[\bdth{b2d_TPd_temp_addictivity}] $\thbfobdmap \nvdash \TPd(x,y,t) \land \TPd(x,y,u) \land \SUMd(s,t,u) \to \TPd(x,y,s)$ 
\hfill{\refdolceax{TPd_temp_addictivity}} 

\item[\bdth{b2d_PCd_temp_addictivity}] $\thbfobdmap \nvdash \PCd(x,y,t) \land \PCd(x,y,u) \land \SUMd(s,t,u) \to \PCd(x,y,s)$ 
\hfill{\refdolceax{PCd_temp_addictivity}} 

\item[\bdth{b2d_TPd_up_slc}] $\thbfobdmap \nvdash \TPd(x,y,t) \land \SLCd(x,s,t) \to \exists r (\SLCd(y,r,t))$ 
\hfill{\refdolceax{TPd_up_slc}} 
\eflist

The same problem affects axioms\nb{CM: questi non sono axioms} like \refbdth{b2d_TPd_temp_addictivity}  and \refbdth{b2d_PCd_temp_addictivity} where from the holding of a given relation at two times, one wants to infer the  holding of the relation at the sum of the times (provided such sum exists). In these cases, the fact that the entity does not exist at the sum of the times rules out (for the majority of relations) the possibility for the relation to hold at the sum. The {\bfo} relation $\bfosregof$ suffers this problem, i.e., a continuant can occupy spatial regions (even the same spatial region) at two given times without having a spatial location at the sum of these times (see the counterexample for \refbdth{b2d_TPd_up_slc}). %In these cases we would not have the spatial location at the sum of the times. 
The presence of spatiotemporal regions and $\bfotproj/\bfosproj$ relations together with the intricate links between $\bfosregof$, $\bfoopart$, $\bfocpart$, $\bfolocated$, and $\bfooccurs$ further complicate the domain of space and spatial location (at least from the formal perspective) impacting all the theorems involving $\SLCd$.

This problem suggests a different mapping technique. Rather than encapsulating the relation between $\bfoexist$ and $\TLCd$ into the mapping  \refbddf{b2d_TLCd}, one could follow a `two steps' procedure: first introduce a direct mapping between $\bfoexist$ and $\PREd$ and then define $\TLCd$ by using $\PREd$ (as defined in the mapping), e.g.,
%
\bflist
\item[--] $\PREd(x,t) \sdef (\PDdcat(x) \lor \EDdcat(x) \lor \Qdcat(x)) \land \bfoexist(x,t)$
\item[--] $\TLCd(x,t) \sdef \forall u(\PREd(x,u) \ifif \Pd(u,t))$
\eflist
%  
In this way, entities that, for instance, exist at two times but not at their mereological sum would still lack a temporal location $\TLCd$ but this would not prevent them to $\PREd$-exist (just relying on the first mapping). We can then hope to recover some theorems that does not require $\TLCd$ at the price of possibly losing the original link between $\PREd$ and $\TLCd$. The study of how this mapping technique can be extended to other primitives and its actual impact are left for future work.   


The problems highlighted up to this point have important impacts but are primarily of technical nature. Some genuine ontological differences seem also to exist. First, the defined $\Pd$, like $\bfoopart$, does not satisfy the supplementation axiom, see \refbdth{b2d_PdSupp}. Second, the relation $\bfoinh$ is not equivalent to the original $\DQTd$, in particular it does not satisfy the non-migration principle \refbdth{b2d_DQTdUnicity}. For relational qualities this can be problematic if, for instance, one wants to distinguish ``Mary loves John'' from ``John loves Mary'': in both cases we would have a relational quality inhering in both John and Mary. Third, the reflexivity of the defined $\TPd$, like the one of $\bfocpart$, does not hold for all the endurants/continuants. It follows that in {\bfo} one can have continuants for which $\bfocpart$ is not defined.  Fourth, while in {\dolce} all the perdurants have participants at every time they exist \refbdth{b2d_PCd_existence_participant}, in {\bfo}, processes need to have at least a participant and this does not hold for process boundaries. Furthermore, in {\bfo}, continuants do not necessarily participate in processes while in {\dolce} this is mandatory: all the endurants participate at a process when they exist \refbdth{b2d_PCd_existence_perdurant}.
%
\bflist
\item[\bdth{b2d_PdSupp}] $\thbfobdmap \nvdash (\ABdcat(x) \lor \PDdcat(x)) \land \neg\Pd(x,y) \to \exists z (\Pd(z,x) \land \neg \Od(z,y))$ 
\hfill{\refdolceax{PdSupp}} 

\item[\bdth{b2d_DQTdUnicity}] $\thbfobdmap \nvdash \DQTd(x,y) \land \DQTd(x,z) \to y=z$ 
\hfill{\refdolceax{DQTdUnicity}} 

\item[\bdth{b2d_PCd_existence_participant}] $\thbfobdmap \nvdash \PDdcat(x) \land \PREd(x,t) \to \exists yu(\Pd(u,t) \land \PCd(y,x,u))$ 
\hfill{\refdolceax{PCd_existence_participant}} 

\item[\bdth{b2d_PCd_existence_perdurant}] $\thbfobdmap \nvdash \EDdcat(x) \land \PREd(x,t) \to \exists yu (\Pd(u,t) \land \PCd(x,y,u))$ 
\hfill{\refdolceax{PCd_existence_perdurant}} 
\eflist

%============================================
\section{Inferring OWL subclass mappings}
%============================================
{\color{red} metti qualche cosa anche vedendo presentazione per Ontocommons oppure togli, non so}

%============================================
\section{Conclusion}
%============================================
 add conclusions



%=================================================

\begin{acks}
ontocommons etc.
\end{acks}

\nocite{label} 
\bibliographystyle{ios2-nameyear}
\bibliography{states&levels}

\end{document}








